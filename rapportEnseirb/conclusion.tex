\chapter*{Conclusion}
Le but de ce stage était de fournir une méthode théorique permettant de répartir l'exécution de réseaux de Petri, de manière à accroître les possibilités créatrices des utilisateurs du logiciel \brand{i-score}. Après une revue des dernières avancées en répartition et en réseaux de Petri, différentes méthodes présentant chacune leurs avantages et inconvénients ont été présentées. Un travail plus important a été effectué sur une de ces méthodes, en essayant de prouver au mieux sa justesse, et en essayant de voir toutes les possibilités qu'elle offre aux compositeurs et scénaristes. Des essais ont été faits avec une estimation fixe de latence, et ont offert des résultats convaincants lorsque les périphériques présentaient une latence proche de l'estimation.

Il reste néanmoins beaucoup de portes ouvertes : ainsi, il serait intéressant de tester plus en avant les deux autres méthodes proposées. Dans la seconde méthode présentée, il serait aussi pertinent de déterminer une notation permettant de réunir les nouveaux segments introduits avec le segment originel dont ils découlent, car ils sont fortement corrélés.
Il reste aussi à effectuer des tests sur un cas à grande échelle (plus de deux ou trois périphériques, et un scénario plus complexe), ce qui permettrait peut-être en échangeant avec un compositeur de trouver de nouveaux usages à la répartition. Enfin, au niveau de l'implémentation, il est nécessaire de rendre la gestion de la latence dynamique pour s'adapter aux différents périphériques pouvant se connecter.
Il est à noter que le travail effectué pendant ce stage pourra être potentiellement être utilisé et continué en cas d'un probable départ en thèse par la suite.