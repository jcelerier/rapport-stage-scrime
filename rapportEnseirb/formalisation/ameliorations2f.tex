\section{Améliorations possibles sur seconde formalisation}
\subsection{Politiques d'\texorpdfstring{$\alpha$-transitions}~}
\label{section:canon}
La sémantique des $\alpha$-transitions peut être modifiée afin de fournir plusieurs politiques.

Par exemple : 
\begin{itemize}
\item Envoyer un jeton vers toutes ses branches en sortie.
\item N'envoyer un jeton que vers certaines branches choisies.
\item Envoyer des jetons vers une ou des branches choisies au hasard.
\item Envoyer les jetons en différé, pour créer un effet de canon par exemple.
\item Garder le fait qu'un jeton soit passé en mémoire et le renvoyer plus tard (pour une reprise sur erreur, par exemple).
\end{itemize}

Cela offre plus de latitude au compositeur. Néanmoins le réseau de Petri n'est plus "pur", on modifie l'algorithme d'exécution d'une transition.


\subsection{Rétropropagation du buffering}
\label{section:retropropag}
Un processus a été conçu pour essayer de régler le problème des latences trop grandes entre nœuds.

Il est utile dans le cas où l'on n'a pas de point d'interaction entre différents segments successifs.

\begin{mydef}[Point d'interaction]
	Un point d'interaction est un endroit où l'exécution du réseau de Petri est soumise à influence extérieure : condition, déclenchement interactif, etc. Une définition complète est donnée dans \cite[p. 148]{allombert2009aspects}.
\end{mydef}

\paragraph{Processus}
Soient deux segments $p$ et $s$ tels que $s$ soit successif à $p$. On désire déplacer $s$, mais la latence entre les deux nœuds est supérieure à la durée de la transition de $p$ : la différence $t_p - \delta$ est négative.


Pour tout segment $u \in \pre(p)$ : 
\begin{enumerate}
\item S'il n'existe pas, créer un $\alpha$-segment le précédant.
\item Rajouter une place en sortie de chaque $\alpha$-segment créé à l'étape précédente.
\item Supprimer l'arc précédant le $\beta$-segment correspondant à $s$ créé lors de \texttt{deplacement(p, s, \dots)}.
\item Créer un arc entre toutes les places créées et ce $\beta$-segment.
\end{enumerate}
Soit $t_p - \delta$ la durée du $\beta$-segment. En raison de la latence élevée, $t_p - \delta$ est négatif. Soient $t_{u_1}, ..., t_{u_n}$ les durées des segments de $\pre(p)$.

On remplace $t_p - \delta$ par $\max(t_{u_1}, ..., t_{u_n}) + t_p - \delta$.

Si la durée obtenue est toujours négative, on peut réitérer ce processus jusqu'à une de ces deux possibilités : 

\begin{itemize}
\item Obtention d'une durée positive ou nulle.
\item Arrivée à un point d'interactivité.
\end{itemize}

\subsection{Utilisation de la duplication pour un backup}
Le backup consiste à être en possession de plusieurs nœuds exécutant la même partie du réseau, mais un seul exécute réellement les processus contenus entre les places.

Cela permet, si un des nœuds rencontre un problème informatique, d'avoir un autre nœud qui prenne le relai.

Deux approches sont possibles :
\begin{itemize}
\item Une, déjà évoquée, est d'avoir l'opération qui repart depuis le début du déplacement ($\alpha$-segment).
\item L'autre serait d'essayer de repartir au plus proche, c'est-à-dire :  
\begin{itemize}
\item Soit replacer un jeton au début de la dernière transition s'étant exécutée. Cela implique un léger retour en arrière.
\item Soit attendre que la dernière transition qui était en cours d'exécution se termine. Cela implique une impression de "mise en pause".
\end{itemize}
\end{itemize}

Dans les deux cas présentés ci-dessus, il sera impossible d'échapper à un effet de saut lors de la reprise pour certains processus. Mais une granularité plus fine (c'est-à-dire au niveau des processus et du temps réel) n'est pas possible avec le formalisme des réseaux de Petri.

Cependant, il est simple de l'implémenter : il suffit d'effectuer une duplication au niveau du réseau de Petri, et de donner aux nœuds du graphe une propriété avec comme états posibles \textit{activé} ou \textit{désactivé}. Si un nœud ne répond plus et est dupliqué à des fins de backup, un autre nœud \textit{active} la branche censée prendre le relai; ce nœud émet alors instantanément des signaux à ses processus.

De plus, l'idéal serait de pouvoir mettre en sourdine (\textit{to mute}) uniquement les sorties matérielles de l'ordinateur (ex. : sortie audio, vidéo-projecteur). En effet, un processus externe peut utiliser un effet de feedback; une transition sans interruptions entre une machine ayant planté et la machine de backup ne fonctionnera correctement que si toutes les étapes précédentes ont été calculées par la machine de backup.

Enfin, pour des questions d'ergonomie, il serait intéressant d'avoir un concept de groupes de nœuds, censés posséder les mêmes tâches, et d'associer à chaque segment un ou plusieurs groupes. La détection d'un nœud "mort" se ferait dans le groupe par envoi de messages réguliers, et une des machines au hasard du groupe reprendrait les tâches laissées en plan.

\subsection{Cas du changement de vitesse}
Le formalisme~\cite{allombert2010virage} permet de changer la vitesse d'exécution, en multipliant les durées des transitions par un facteur numérique.

Pour conserver la correction du formalisme décrit ici, il ne faut pas multiplier le facteur de décalage entre les nœuds, car il ne dépend pas de la vitesse d'exécution du morceau: il s'agit d'une caractéristique physique du réseau.

Il serait par exemple possible d'affecter une fonction affine de la forme $av + b$ aux transitions, ou $a$ est la durée de la transition canonique, $v$ le coefficient multiplicateur de vitesse compris entre $0.5$ et $2$, et $b$ le délai. 

\subsection{Simplifications}
\subsubsection{$\alpha$-transitions}
\begin{itemize}
\item Si un $\alpha$-segment est en sortie d'un autre, ils peuvent être fusionnés : les places de sortie du segment suivant sont reliés à la transition du précédent. Cela ne change rien : dans les deux cas les délais des transitions sont nuls.

\item De même, si un $\alpha$-segment suit directement un $\beta$-segment, on peut supprimer l'$\alpha$-segment et transférer les places qui étaient à sa sortie, en sortie du $\beta$-segment.
\end{itemize}
\subsubsection{Connaissance de la topologie du réseau de Petri}
On peut essayer de faire en sorte que les nœuds n'aient connaissance que des segments qu'ils possèdent et sont censés exécuter, et ce afin d'éviter la duplication.

Ils doivent au moins posséder : 
\begin{itemize}
\item Leurs places d'entrée, pour savoir quand les transitions sont sensibilisées.
\item Leurs transitions.
\item Leurs places de sortie qui correspondent à une fin de processus.
\end{itemize}

Les autres envois de jetons peuvent être faits par le réseau.

\paragraph{Prise de connaissance dynamique ?}
Tant que l'on se situe avant l'exécution d'un segment par une machine, une autre machine peut lui transférer ce segment par le réseau.

Il faut néanmoins s'assurer qu'elle reçoive le / les segments avant leurs débuts, en calculant la latence entre les nœuds, et informer les autres nœuds du nouveau responsable de ce segment, pour qu'ils sachent ou envoyer leurs jetons.

\subsubsection{Connaissance de l'état du réseau de Petri}
Si l'on opte pour une connaissance globale, est-il néanmoins nécessaire que chaque nœud ait connaissance de la position de chaque jeton à chaque instant et fasse évoluer ses propres jetons de son côté ? 

Cela pose problème dans :
\begin{itemize}
\item Le cas des points d'interaction, ce qui engendre parfois une surcharge au niveau du réseau, si par exemple la valeur d'une variable sur laquelle porte une condition doit être transmise à tous les nœuds.
\item Le cas où une machine aurait un ralentissement : il y aurait une incohérence entre la position de son jeton et la position du même jeton dans les autres machines.
\end{itemize}

Une solution serait d'avoir une machine maîtresse, dont l'état est prioritaire sur tous les autres. 

Ou alors de ne pas avoir de réplication d'état : cela veut dire que la plupart des machines seront toujours dans un état incohérent vis-à-vis de la sémantique des réseaux de Petri (car des jetons vont apparaître et disparaître au gré des envois par le réseau), mais le fonctionnement est plus simple à gérer et provoquerait moins de problèmes sur le long terme.

\subsection{Déplacement de groupes de segments}
Si on désire déplacer plusieurs segments adjacents en même temps, il suffit d'appliquer l'opération de déplacement aux segments possédant des places d'entrée ou de sortie communes à un segment ne faisant pas partie de ce groupe. En effet, seuls ceux-là sont sensibles au délai entre les nœuds.

\subsection{Approche optimisée pour implémentation}
Comme i-score précompile les réseaux de Petri à chaque lecture, et ne les génère pas de façon dynamique, il serait possible d'appliquer le processus "inverse" de celui décrit, pour avoir un nombre de calculs minimaux à réaliser : 

\begin{enumerate}
\item Affecter à chaque segment le nœud sur lequel il est censé se trouver.
\item Pour chaque paire de segments adjacents et n'étant pas sur le même nœud, effectuer une opération de déplacement.
\end{enumerate}

Sans cette optimisation, il y aurait par exemple des cas de déplacements suivis de recombinaisons inutiles.