\section{\scshape Introduction}
\subsection{Composition}
\begin{frame}
	\frametitle{Deux modalités de composition}
	\begin{block}<+->{Orientée partition}
		\begin{itemize}
			\item Pièce sur support
			%\item Métaphore de la flêche
			\item Time-line
			\item Bloc univers
			\item Déterminisme
		\end{itemize}
	\end{block}
	
	\begin{block}<+->{Orientée performance}
		\begin{itemize}
			\item Programme temps réel
			%\item Métaphore du fleuve
			\item Flots temporisés de données (streams)
			\item Cours du temps
			\item Indéterminisme
		\end{itemize}
	\end{block}
\end{frame}


\begin{frame}
	\frametitle{Modélisation hybride du temps}
	\begin{block}<+->{Les séries de Mc Taggart}
		\begin{itemize}
			\item Série A~: être passé, être présent, être futur;
			\item Série B~: être avant, être pendant, être après;
			\item Série C~: chronologique
		\end{itemize}
	\end{block}
	
	\begin{block}<+->{Supports temporels}
		\begin{itemize}
			\item Time-flow~: série A, événements dynamiques;
			\item Time-line~: série B, événements statiques;
			\item Temps granulaire~: série C, tout événement;
		\end{itemize}
	\end{block}
\end{frame}

\subsection{Partitions interactives}
\begin{frame}
	\frametitle{Partition interactive}
	\begin{block}<+->{Interpétation}
		\begin{itemize}
			\item Espace de liberté pour le musicien
			\item Limites imposées par le compositeur
			\item Plusieurs pièces musicales possibles
			\item Articulations, modifications agogiques
		\end{itemize}
	\end{block}
	
	\begin{block}<+->{Objets temporels}
		\begin{itemize}
			\item Événements
			\begin{itemize}
				\item Statiques : datés sur la time-line
				\item Dynamiques : datés à l'exécution
			\end{itemize}
			\item Intervalles
			\item Boîtes processus
			\item Boîtes hiérarchiques
		\end{itemize}
	\end{block}
\end{frame}

\begin{frame}
	\frametitle{Partition interactives}
	\begin{block}<+->{Principe de construction}
		\begin{itemize}
			\item Soustractive / additive
			\item Incertitude positions et durées
			\item Contraintes temporelles~: logique de points
			\item Restriction de l'ensemble des possibles
			\item Maintien automatique de la cohérence de la pièce
		\end{itemize}
	\end{block}
	
	\begin{block}<+->{Principe d'exécution}
		\begin{itemize}
			\item Cohérence~: relations temporelles
			\item Ordre partiel de positions
			\item Modèle concurrent de processus
			\item Réseaux de Petri, NTCC, RML
		\end{itemize}
	\end{block}
\end{frame}

\begin{frame}
	\frametitle{Extension non linéaire}
	\begin{block}<+->{Événements conditionnés}
		\begin{itemize}
			\item Généralisation des événements dynamiques
			\item Certains événements ne s'exécutent pas
			\item Plusieurs branches possibles
			\item \emph{Amplitude d'existence}~: portées par les jetons
		\end{itemize}
	\end{block}
	
	\begin{block}<+->{Boucles}
		\begin{itemize}
			\item Généralisation des intervalles
			\item Temps lisse / temps strié
			\item Certains événements s'exécutent plusieurs fois
			\item Répétition
		\end{itemize}
	\end{block}
\end{frame}

\begin{frame}{Support théorique}
	Réseaux de Petri : outil de modélisation de processus, conçu dans les années 60.
\end{frame}

\subsection{OSSIA}
\begin{frame}{OSSIA}
	Implémentation des concepts théoriques permettant leur utilisation par les compositeurs.
	
	Notions : 
	\begin{itemize}
		\item{Temps:
		\begin{itemize}
			\item Scénario.
			\item Évènement.
			\item TimeBox.
			\item TimeProcess.
		\end{itemize}
		}
		
		\item{Interopérabilité:
		\begin{itemize}
			\item Jamoma.
			\item Minuit.
		\end{itemize}
	}
	\end{itemize}
	
	\begin{itemize}
		\itemar Implémentation partielle dans i-score~0.2.
		\itemar Implémentation totale visée pour i-score~0.3.
	\end{itemize}
\end{frame}