\section{\scshape Répartition}
\subsection{Problématique}
\begin{frame}
	\frametitle{Problématique}
	\begin{block}<+->{Une seule machine}
		\begin{itemize}
			\item Horloge globale
			\item Hypothèse synchrone
			\item Ordre des calculs dans chaque cycle
			\item Durées exprimées en cycles
		\end{itemize}
	\end{block}
	
	\begin{block}<+->{Plusieurs machines}
		\begin{itemize}
			\item Horloges locales
			\item Relativité : échanges de signaux
			\item Distances temporelles : durée de transmission
			\item Topologie du réseau
		\end{itemize}
	\end{block}
\end{frame}
\subsection*{Présentation}
\begin{frame}{Présentation}
	{\large Répartition des réseaux de Petri dans i-score}
	
	Aspects principaux : 
	\begin{itemize}
		\item Réseaux de Petri $\rightarrow$ recherche d'existant et étude théorique.
		\item i-score $\rightarrow$ implémentation.
	\end{itemize} 
	\vspace{1em}
	
	Fonctionnalités désirées : 
	\begin{itemize}
		\item Édition collaborative.
		\item Fonctionnement avec les concepts \textsc{OSSIA}.
		\item Fonctionnement sur plate-formes embarquées (smartphones, \textsc{Raspberry Pi}, \textsc{BeagleBoard}).
		\item Multiples méthodes de synchronisation.
		\item Backup, mode canon.
	\end{itemize}
\end{frame}

\subsection{Synchronisation}
\begin{frame}{Répartition et synchronisation}
	Le problème n'est pas un problème de répartition algorithmique:
	le compositeur place lui-même les scénarios en fonction des logiciels dont il dispose.
	
	~ \\
	Deux aspects de la répartition désirée : 
	\begin{itemize}
		\item Déplacement de scénarios.
		\item Duplication de scénarios. 
	\end{itemize}
	~ \\

	Méthode générale retenue : 
	\begin{itemize}
		\itemar Estimer la latence entre les paires de machines communiquantes, et faire partir les scénarios plus tôt en utilisant cette information.
		\itemar Implémentation en terme de réseaux de Petri, mais aussi implémentation générique dans \textsc{OSSIA}.
	\end{itemize}
\end{frame}
