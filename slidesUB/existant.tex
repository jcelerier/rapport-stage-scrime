
%%%%%%%%%%%%%%%%%%%%%%%%%%%%%%%%%%%%%%%%%%%%%%%%%%%%%%
%%%%%%%%%%%%%%%%%%%%%%%%%%%%%%%%%%%%%%%%%%%%%%%%%%%%%%
\section{\scshape Théorisation}
\subsection{Réseaux de Petri}
\begin{frame}{Utilisation des réseaux de Petri}
Utilisé comme moteur d'exécution : traduction des scénarios en réseaux de Petri [citer allombert].

Autres moteurs en développement : Reactive ML, ...

Multiples possibilités se chevauchant :
\begin{itemize}
	\item Réseaux colorés.
	\item Réseaux temporels et temporisés.
	\item Réseaux spatio-temporels.
	\item Réseaux hiérarchiques.
\end{itemize}

\end{frame}


%%%%%%%%%%%%%%%%%%%%%%%%%%%%%%%%%%%%%%%%%%%%%%%%%%%%%%
%%%%%%%%%%%%%%%%%%%%%%%%%%%%%%%%%%%%%%%%%%%%%%%%%%%%%%
\subsection{Répartition}
\begin{frame}{Répartition et synchronisation}
	Le problème n'est pas un problème de répartition algorithmique.
	\begin{itemize}
		\itemar Le compositeur place lui-même les scénarios en fonction des logiciels dont il dispose.
	\end{itemize}
	
	Deux aspects à la répartition désirée : 
	\begin{itemize}
		\item Déplacement.
		\item Duplication. 
	\end{itemize}
	~ \\
	Nécessaire : estimation de latence.
	
	Utile : synchronisation en temps des machines.
	\begin{itemize}
		\itemar{Protocoles PTP, NTP\dots}
	\end{itemize}
\end{frame}
