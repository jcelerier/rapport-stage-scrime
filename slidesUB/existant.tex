\section{\scshape Théorisation}
\subsection{Réseaux de Petri}
\begin{frame}{Utilisation des réseaux de Petri}
Utilisé comme moteur d'exécution : traduction des scénarios en réseaux de Petri\cite{allombert2009aspects}.

Autres moteurs possibles : Reactive ML\cite{arias2014modelling}, NTCC\cite{allombert2006concurrent}.

Multiples possibilités se chevauchant :
\begin{itemize}
	\item Réseaux colorés.
	\item Réseaux temporels et temporisés.
	\item Réseaux spatio-temporels.
	\item Réseaux hiérarchiques.
\end{itemize}
\end{frame}

\subsection{Répartition}
\begin{frame}{Répartition et synchronisation}
	Le problème n'est pas un problème de répartition algorithmique:
	Le compositeur place lui-même les scénarios en fonction des logiciels dont il dispose.
	
	~ \\
	Deux aspects à la répartition désirée : 
	\begin{itemize}
		\item Déplacement de scénarios.
		\item Duplication de scénarios. 
	\end{itemize}
	~ \\
	Nécessaire : estimation de latence pour les faire partir au bon moment.
	\begin{itemize}
		\itemar Méthodes par ping, P2P\cite{im2000method}, King\cite{gummadi2002king}.
	\end{itemize}
	
	Peut simplifier : synchronisation en temps des machines.
	\begin{itemize}
		\itemar{Protocoles \textsc{PTP}\cite{scheiterer2009synchronization}, \textsc{NTP}\cite{mills1991internet}}
	\end{itemize}
\end{frame}
