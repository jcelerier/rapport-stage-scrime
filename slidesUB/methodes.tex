\subsection{Approche suivie}
\begin{frame}{Approche suivie}
	\begin{itemize}
		\itemar Introduire la notion de répartition par des modifications du réseau de Petri.
		\itemar Élément atomique utilisé : \textit{segment}.
	\end{itemize}

	Trois méthodes proposées.
	\begin{itemize}
		\item Deux modifient la topologie du réseau de Petri.
		\item Une modifie la sémantique du réseau de Petri.
	\end{itemize}
	
	Une méthode par modification de la topologie a été retenue pour une étude plus poussée.
\end{frame}
\subsection{Première méthode}
\begin{frame}
	\begin{figure}[h!]
		\centering
		On déplace s sur un nœud $N_d$ : 

\begin{tikzpicture}[node distance=1.3cm,>=stealth,bend angle=45,auto]

  \tikzstyle{segment}=[]
  \tikzstyle{place}=[circle,thick,draw=blue!75,fill=blue!20,minimum size=6mm]
  \tikzstyle{red place}=[place,draw=red!75,fill=red!20]
  \tikzstyle{transition}=[rectangle,thick,draw=black!75,
  			  fill=black!20,minimum size=4mm]
  			  
  \tikzstyle{black box}=[draw=black, fill=black!30, draw opacity=0.7, fill opacity=0.2, thick, dashed, inner sep=0.5cm]


  \begin{scope}
    \node [transition] (Tk) [label=$t_p$] {};
	\node [segment] (Pk) [left of=Tk] {$\pre(p)$};
	\node [segment] (S) [right of=Tk] {$s$};
	\node [segment] (Sk) [below of=S] {$\post(p) \setminus s$};
			
  	\path
  		(Pk) edge[post] node {} (Tk)
  		(Tk) edge[post] node {} (S)
  			 edge[post] node {} (Sk);
  
   \begin{pgfonlayer}{background}
    \node [black box, fit=(Tk) (Pk) (S) (Sk)] (Box1) {};
    \node[above right] at (Box1.south west) {Nœud $N$};
   \end{pgfonlayer}
  \end{scope}
  
  
\end{tikzpicture}

\vspace{1em}
Devient
\vspace{1em}

\begin{tikzpicture}[node distance=1.3cm,>=stealth,bend angle=45,auto]
 
   \tikzstyle{segment}=[]
   \tikzstyle{place}=[circle,thick,draw=blue!75,fill=blue!20,minimum size=6mm]
   \tikzstyle{red place}=[place,draw=red!75,fill=red!20]
   \tikzstyle{transition}=[rectangle,thick,draw=black!75,
   fill=black!20,minimum size=4mm]
   
    \tikzstyle{black box}=[draw=black, fill=black!30, draw opacity=0.7, fill opacity=0.2, thick, dashed, inner sep=0.7cm]
    \tikzstyle{red box}=[draw=red, minimum size=2.1cm, fill=red!30, draw opacity=0.7, fill opacity=0.2, thick, dashed, inner sep=0.7cm]
 
  \begin{scope}
	\node [transition] (Tk) [label=$t_p - \delta$] {};
  	\node [segment] (Pk) [left of=Tk] {$\pre(p)$};
  	\node [segment] (S) [right of=Tk] at (1.5, 2) {$s$};
    
    \node [place] (PN) at (1, -1.3){};
    \node [transition] (TN) [right of=PN,  label=$\delta$] {};
    \node [segment] (Sk) [right of=TN] at (3, -1.3) {$\post(p) \setminus s$};
    \node [segment] (C) [below of=Sk] {$\memory(s, p)$};
    
    \path
      (Pk) edge[post] node {} (Tk)
      (Tk) edge[post] node {} (S)
           edge[post] node {} (PN)
      (PN) edge[post] node {} (TN)
      (TN) edge[post] node {} (Sk)
      	   edge[post] node {} (C); 
    	
    \begin{pgfonlayer}{background}
    \node [black box, fit=(Tk) (Pk) (Sk) (PN) (TN) (C)] (Box1) {};
    \node[above right] at (Box1.south west) {Nœud $N$};
    
     \node [red box, fit=(S)] (Box2) {};
     \node[above left] at (Box2.south east) {Nœud $N_d$};
     \end{pgfonlayer}
   \end{scope}
   
\end{tikzpicture}

~ \\ 
Maintenant, $s$ peut être exécuté sur $N_d$ : le délai de transmission est pris en compte.
  
		\caption{Première méthode, avant déplacement}
		\label{fig:deplacementMethode1}
	\end{figure}
\end{frame}

\begin{frame}
	\begin{figure}[h!]
		\centering
		\begin{tikzpicture}[node distance=1.3cm,>=stealth,bend angle=45,auto]
 
   \tikzstyle{segment}=[]
   \tikzstyle{place}=[circle,thick,draw=blue!75,fill=blue!20,minimum size=6mm]
   \tikzstyle{red place}=[place,draw=red!75,fill=red!20]
   \tikzstyle{transition}=[rectangle,thick,draw=black!75,
   fill=black!20,minimum size=4mm]
   
    \tikzstyle{black box}=[draw=black, fill=black!30, draw opacity=0.7, fill opacity=0.2, thick, dashed, inner sep=0.7cm]
    \tikzstyle{red box}=[draw=red, minimum size=2.1cm, fill=red!30, draw opacity=0.7, fill opacity=0.2, thick, dashed, inner sep=0.7cm]
 
  \begin{scope}
	\node [transition] (Tk) [label=$t_p - \delta$] {};
  	\node [segment] (Pk) [left of=Tk] {$\pre(p)$};
  	\node [segment] (S) [right of=Tk] at (1.5, 2) {$s$};
    
    \node [place] (PN) at (1, -1.3){};
    \node [transition] (TN) [right of=PN,  label=$\delta$] {};
    \node [segment] (Sk) [right of=TN] at (3, -1.3) {$\post(p) \setminus s$};
    \node [segment] (C) [below of=Sk] {$\memory(s, p)$};
    
    \path
      (Pk) edge[post] node {} (Tk)
      (Tk) edge[post] node {} (S)
           edge[post] node {} (PN)
      (PN) edge[post] node {} (TN)
      (TN) edge[post] node {} (Sk)
      	   edge[post] node {} (C); 
    	
    \begin{pgfonlayer}{background}
    \node [black box, fit=(Tk) (Pk) (Sk) (PN) (TN) (C)] (Box1) {};
    \node[above right] at (Box1.south west) {Nœud $N$};
    
     \node [red box, fit=(S)] (Box2) {};
     \node[above left] at (Box2.south east) {Nœud $N_d$};
     \end{pgfonlayer}
   \end{scope}
   
\end{tikzpicture}
		\caption{Première méthode, après déplacement}
		\label{fig:deplacementMethode12}
	\end{figure}
\end{frame}