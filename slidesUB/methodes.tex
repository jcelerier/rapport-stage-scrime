\subsection{Approche suivie}
\begin{frame}{Approche suivie}
	\begin{itemize}
		\itemar Introduire les notions de répartition désirées par des modifications du réseau de Petri.
		\itemar Prendre en compte la latence.
		\itemar Élément atomique utilisé : \textit{segment}.
	\end{itemize}
	\vspace{1em}
	Trois méthodes proposées.
	\begin{itemize}
		\item Deux modifient la topologie du réseau de Petri.
		\item Une modifie la sémantique du réseau de Petri.
	\end{itemize}
	
	Une méthode par modification de la topologie a été retenue pour une étude plus poussée.
\end{frame}

\section{\scshape Méthodes}
\subsection{Première méthode}
\begin{frame}{Première méthode}
	\begin{figure}[h!]
		\centering
		\resizebox{!}{2cm}{\begin{tikzpicture}[node distance=1.3cm,>=stealth,bend angle=45,auto]

  \tikzstyle{segment}=[]
  \tikzstyle{place}=[circle,thick,draw=blue!75,fill=blue!20,minimum size=6mm]
  \tikzstyle{red place}=[place,draw=red!75,fill=red!20]
  \tikzstyle{transition}=[rectangle,thick,draw=black!75,
  			  fill=black!20,minimum size=4mm]
  			  
  \tikzstyle{black box}=[draw=black, fill=black!30, draw opacity=0.7, fill opacity=0.2, thick, dashed, inner sep=0.5cm]


  \begin{scope}
    \node [transition] (Tk) [label=$t_p$] {};
	\node [segment] (Pk) [left of=Tk] {$\pre(p)$};
	\node [segment] (S) [right of=Tk] {$s$};
	\node [segment] (Sk) [below of=S] {$\post(p) \setminus s$};
			
  	\path
  		(Pk) edge[post] node {} (Tk)
  		(Tk) edge[post] node {} (S)
  			 edge[post] node {} (Sk);
  
   \begin{pgfonlayer}{background}
    \node [black box, fit=(Tk) (Pk) (S) (Sk)] (Box1) {};
    \node[above right] at (Box1.south west) {Nœud $N$};
   \end{pgfonlayer}
  \end{scope}
\end{tikzpicture}

  }
		\caption{Avant déplacement}
		\label{fig:deplacementMethode1}
	\end{figure}
	\begin{figure}[h!]
		\centering
		\resizebox{!}{3cm}{\input{figures/deplacementMethode1suite}}
		\caption{Après déplacement}
		\label{fig:deplacementMethode1-2}
	\end{figure}
\end{frame}

\subsection{Seconde méthode}
\begin{frame}{Seconde méthode}
	\begin{figure}[h!]
		\centering
		\begin{tikzpicture}[node distance=1.3cm,>=stealth,bend angle=45,auto]
  \tikzstyle{segment}=[]
  \tikzstyle{place}=[circle,thick,draw=blue!75,fill=blue!20,minimum size=6mm]
  \tikzstyle{red place}=[place,draw=red!75,fill=red!20]
  \tikzstyle{transition}=[rectangle,thick,draw=black!75,
  			  fill=black!20,minimum size=4mm]


  \begin{scope}
  \node[segment](I){In};
  \node [place] (A) [right of=I] {$A_1$};
  \node [place] (A2) [below of=A] {$A_2$};
  \node[segment](I2) [left of=A2]{In 2};
  \node [transition] (Ta) [right of=A, label=$t_A$] {$T_A$};
  \node [place] (B) [right of=Ta] {$B_1$};
  \node [place] (A3) [above of=B] {$A_3$};
  \node [transition] (Tb) [right of=B, label=$t_B$] {$T_B$};
  \node [place] (C) [right of=Tb] {$C_1$};
  \node[segment](O) [right of=C]{Sortie};
  
  \path
  		(I) edge[post] node {} (A)
  		(I2) edge[post] node {} (A2)
  		(A) edge[post] node {} (Ta)
  		(A2) edge[post] node{} (Ta)
  		(Ta) edge[post] node {} (B)
  			 edge[post] node {} (A3)
  		(B) edge[post] node {} (Tb)
  		(Tb) edge[post] node {} (C)
  		(C) edge[post] node {} (O);
  \end{scope}
\end{tikzpicture}
  
\vspace{1em}
Devient
\vspace{1em}
  
\begin{tikzpicture}[node distance=1.3cm,>=stealth,bend angle=45,auto]
  \tikzstyle{segment}=[]
  \tikzstyle{place}=[circle,thick,draw=blue!75,fill=blue!20,minimum size=6mm]
  \tikzstyle{red place}=[place,draw=red!75,fill=red!20]
  \tikzstyle{transition}=[rectangle,thick,draw=black!75,
  			  fill=black!20,minimum size=4mm]
  \tikzstyle{red}=[draw=red!75,fill=red!20]
 \tikzstyle{green}=[draw=green!75,fill=green!20]

  \begin{scope}
  %alpha-sgmt
  \node[segment](I){In};
  \node [place] (Aalpha) [red, right of=I] {$A_1$};
  \node [transition] (Talpha) [red,right of=Aalpha, label=$0$] {$T_{\alpha A}$};

  \node[segment](I2) [left of=A2]{In 2};
  \node [place] (Aalpha2) [red, below of=Aalpha] {$A_2$};
%  \node [transition] (Talpha2) [red,below of=Talpha, label=$0$] {};

  \node [place] (A) [right of=Talpha] {$A_i$};
  
  \node [transition] (Ta) [right of=A, label=$t_A$] {$T_A$};
  \node [place,dotted] (Bprime) [right of=Ta] {$B'_1$};
  \node [place] (A3) [above of=Bprime] {$A_3$};
  \node [place] (AS) [green,below of=A] {$A_B$};
  \node [transition] (TS) [green,right of=AS, label=$t_A - x$] {$T_{\beta B}$};
  
  \node [place] (B) [right of=TS] {$B_1$};
  \node [transition] (Tb) [right of=B, label=$t_B$] {$T_B$};
  \node [place] (C) [right of=Tb] {$C_1$};
  \node[segment](O) [right of=C]{Sortie};
  
  \path
  		(I) edge[post] node {} (Aalpha) 
  		(I2) edge[post] node {} (Aalpha2)
  		(Aalpha) edge[post,red] node {} (Talpha)
  		(Talpha) edge[red,post] node {} (A)
  		         edge[red,post] node {} (AS)
  		(Aalpha2) edge[post,red] node {} (Talpha)
%  		(Talpha2) edge[red,post] node {} (A)
%  		         edge[red,post] node {} (AS)
  		(A) edge[post] node {} (Ta) 
  		(Ta) edge[post] node {} (Bprime) 
  			 edge[post] node {} (A3)
  		(AS) edge[green,post] node {} (TS) 
  		(TS) edge[green,post] node {} (B)         
  		(B) edge[post] node {} (Tb)
  		(Tb) edge[post] node {} (C)
  		(C) edge[post] node {} (O);
  \end{scope}
\end{tikzpicture}
		\caption{Avant déplacement}
		\label{fig:deplacementMethode2}
	\end{figure}
\end{frame}

\begin{frame}{Seconde méthode}
	\begin{figure}[h!]
		\centering
		\resizebox{11cm}{!}{\begin{tikzpicture}[node distance=1.3cm,>=stealth,bend angle=45,auto]
  \tikzstyle{segment}=[]
  \tikzstyle{place}=[circle,thick,draw=blue!75,fill=blue!20,minimum size=6mm]
  \tikzstyle{red place}=[place,draw=red!75,fill=red!20]
  \tikzstyle{transition}=[rectangle,thick,draw=black!75,
  			  fill=black!20,minimum size=4mm]
  \tikzstyle{red}=[draw=red!75,fill=red!20]
 \tikzstyle{green}=[draw=green!75,fill=green!20]
 
    \tikzstyle{black box}=[draw=black, fill=black!30, draw opacity=0.7, fill opacity=0.2, thick, dashed, inner sep=0.7cm]
    \tikzstyle{red box}=[draw=red, minimum size=2.1cm, fill=red!30, draw opacity=0.7, fill opacity=0.2, thick, dashed, inner sep=0.7cm]

  \begin{scope}
  %alpha-sgmt
  \node[segment](I){In};
  \node [place] (Aalpha) [red, right of=I] {$A_1$};
  \node [transition] (Talpha) [red,right of=Aalpha, label=$0$] {};

  \node[segment](I2) [left of=A2]{In 2};
  \node [place] (Aalpha2) [red, below of=Aalpha] {$A_2$};
%  \node [transition] (Talpha2) [red,below of=Talpha, label=$0$] {};

  \node [place] (A) [right of=Talpha] {};
  
  \node [transition] (Ta) [right of=A, label=$t_A$] {$T_A$};
  \node [place,dotted] (Bprime) [right of=Ta] {$B'_1$};
  \node [place] (A3) [above of=Bprime] {$A_3$};
  \node [place] (AS) [green, below of=A]  {};
  \node [transition] (TS) [green,right of=AS, label=$t_A - \delta_{N \rightarrow N_d}$] at (5, -2.3) {};
  
  \node [place] (B) [] at (8, -4.2) {$B_1$};
  \node [transition] (Tb) [right of=B, label=$t_B$] {$T_B$};
  \node [place] (C) [right of=Tb] {$C_1$};
  
  \path
  		(I) edge[post] node {} (Aalpha) 
  		(I2) edge[post] node {} (Aalpha2)
  		(Aalpha) edge[post,red] node {} (Talpha)
  		(Talpha) edge[red,post] node {} (A)
  		         edge[red,post] node {} (AS)
  		(Aalpha2) edge[post,red] node {} (Talpha)
%  		(Talpha2) edge[red,post] node {} (A)
%  		         edge[red,post] node {} (AS)
  		(A) edge[post] node {} (Ta) 
  		(Ta) edge[post] node {} (Bprime) 
  			 edge[post] node {} (A3)
  		(AS) edge[green,post] node {} (TS) 
  		(TS) edge[green,post] node {} (B)         
  		(B) edge[post] node {} (Tb)
  		(Tb) edge[post] node {} (C);
  		
   \begin{pgfonlayer}{background}
    \node [black box, fit=(I2) (A2) (A3) (TS)] (Box1) {};
  	\node [below right] at (Box1.north west) {Nœud $N$};
  	\node [black box, fit=(TS) (C), inner sep=1cm] (Box2) {};
  	\node [above left] at (Box2.south east) {Nœud $N_d$};
   \end{pgfonlayer}
  \end{scope}
\end{tikzpicture}}
		\caption{Après déplacement}
		\label{fig:deplacementMethode2-2}
	\end{figure}
\end{frame}

\subsection{Troisième méthode}
\begin{frame}{Troisième méthode}
	On modifie l'algorithme d'exécution des transitions : on différencie la durée de la transition en fonction de la couleur de la place qui la suit. 
	
	\begin{itemize}
		\itemar Création de correspondances \textit{paire de couleurs} $\leftrightarrow$ \textit{latence}.
		\itemar La transition se fait en plusieurs étapes.
		\itemar Utilisation de la notion d'arcs filtrants et de jetons colorés, pour la duplication.
	\end{itemize}
	
	Pour la duplication, il est de plus nécessaire de changer la pondération des arcs de sortie, et de placer un jeton par couleur de la place de sortie.
\end{frame}

\begin{frame}
	La seconde méthode est retenue.
	
	\begin{itemize}
		\item[\textbullet] Améliorations proposées : 
			\begin{itemize}
				\item Utilisation pour secours informatique (\textit{backup}).
				\item Fonctionnement en mode «~canon~».
				\item Rétro-propagation si délai trop important.
				\item Simplifications de l'algorithme dans les cas courants.
			\end{itemize}
		\vspace{1em}
		\item[\textbullet] Différents modes de synchronisation : 
			\begin{itemize}
				\item Attente de tous.
				\item Duplication de chaque jeton.
				\item Un seul continue.
			\end{itemize}
	\end{itemize}
\end{frame}