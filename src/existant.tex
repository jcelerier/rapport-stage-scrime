\chapter{État de l'art}

\section{Réseaux de Petri}
Les réseaux de Petri (citer thèse C.A. Petri) sont au cœur du formalisme utilisé ici. Néanmoins, une différence majeure entre leur utilisation dans le projet \ac{OSSIA} avec d'autres utilisations plus courante est que pour \ac{OSSIA}, les réseaux ne servent pas à faire de l'analyse statique mais sont au cœur du moteur d'exécution du programme. C'est-à-dire que la théorie des scénarios a été développée de manière à pouvoir être exprimée en terme de réseaux de Petri, et que ces réseaux de Petri sont ensuite exécutés avec un algorithme standard d'exécution, qui sera détaillé plus tard dans ce document.

\subsection{Familles de réseaux de Petri}
% http://sfa.univ-poitiers.fr/informatique-telecom/lib/exe/fetch.php?media=quotidien:m2:ue:spec_temps_reel:extension_rdp_10.pdf
\subsubsection{Base}
État des lieux assez général : \cite{david2010discrete}
\subsubsection{Réseaux hiérarchiques}
Réseaux hiérarchiques par blocs:  \cite{fehling1993concept}.
\subsubsection{Réseaux colorés}
Ouvrage de référence : \cite{jensen1987coloured}
Hierarchies in coloured petri nets : \cite{rozenberg1991advances}
\subsubsection{Réseaux temporels}
Article d'origine : \cite{suzuki1989temporal}
Abstraction au dessus des réseaux temporels : \cite{klai2013temporal} (encode tps écoulé min/max à chaque instant)
\subsubsection{Réseaux stochastiques}
Stochastic Petri Nets, GSPN
\subsubsection{Réseaux flous}
Fuzzy petri net : Dynamic representation of fuzzy knowledge based on fuzzy petri net and genetic-particle swarm optimization
\subsubsection{Réseaux de haut niveau}
HLPN

\subsection{Standardisation}
PNML % http://www.pnml.org/papers/gl93-2010.pdf

\subsection{Outils développés pour l'analyse des réseaux de Petri}
\subsubsection{CPN tools}
\subsubsection{Simulation orientée objet : PNlib}
An object-oriented Petri net simulation tool for hybrid biological processes
\subsubsection{Autres...}

\subsection{Applications récentes}
Il serait très dur d'établir un inventaire complet des utilisations des réseaux de Petri : comme pour les automates, les machines de Turing, ou encore le lambda-calcul, c'est un outil avec une très forte puissance d'expression, qui peut donc être utilisé dans une variété d'environnements différents, à des fins de modélisation.

Par exemple, à l'origine, C. A. Petri avait conçu les réseaux qui portent aujourd'hui son nom afin de modéliser des réactions chimiques.

\subsubsection{Applications à la répartition et à la parallélisation}
* Research on Parallel Test Task Scheduling Based on Improved Genetic Algorithm and Petri Net
\subsubsection{Application aux interfaces homme-système}
* Elementary Events for Modeling of Human-System Interactions with Petri Net Models
\subsubsection{Application dans des modèles physiques}
* Modeling and Analysis for CPS Physical Entities Based on Spatio-Temporal Petri Net.
\subsubsection{Application à la biologie}
* Biologie, etc... (cas original)

* Petri Nets in the Biosciences

\section{Projet OSSIA}
(Présentation travaux Allombert, Toro, Arias, etc.)

Modelling Data Processing for Interactive Scores Using Coloured Petri Nets
\subsection{Les réseaux de Petri dans OSSIA}
\subsection{Scénarios interactifs}
\subsection{Temps souple}
\subsection{Exemple dans le formalisme OSSIA}

\section{Répartition}
\subsection{Ouvrages de référence}
\cite{lynch1996distributed}

\subsection{Développment}
Modularity in the design of robust distributed algorithms

Fully Distributed Algorithms for Minimum Delay Routing Under Heavy Traffic 

A Logic-Based Framework for Verifying Consensus Algorithms

Message-Passing Algorithms for the Verification of Distributed Protocols

Exploiting wireless broadcast property to improve performance of distributed algorithms and mac protocols in wireless networks

Distributed Mutual Exclusion Algorithms for Intersection Traffic Control

New Techniques and Tighter Bounds for Local Computation Algorithms (pertinence bof)

\section{Synchronisation}

\subsection{Synchronisation logique}
\subsubsection{Origines : les horloges de Lamport}
\subsubsection{Horloges vectorielles}

\subsection{Synchronisation physique}
\subsubsection{Algorithmes couramment utilisés}
\paragraph{NTP}
\ac{NTP}
\paragraph{PTP}
\ac{PTP}
Cf. articles sur précision sur Android : 
A Measurement of Time synchronization on Mobile Devices.

Comparer avec NTP + implémentation existantes.

\subsubsection{Travaux plus récents}
(TODO mieux catégoriser)
Distributed synchronization under uncertainty: A fuzzy approach

Fine-grained network time synchronization using reference broadcasts

Subscription-Notification mechanisms for synchronization of distributed states

One Clock to Rule Them All: A Primitive for Distributed Wireless Protocols at the Physical Layer

Distributed network synchronization: the internet and electric power grids

Riccati Design for Synchronization of Discrete-Time Systems (bof)

Riccati Design for Synchronization of Continuous-Time Systems

Time Synchronization for High Latency Acoustic Networks


\subsection{Estimation de la latence}
\label{section:latence}
Network latency estimation % http://www.google.com/patents/US8254264B1

King: estimating latency between arbitrary internet end hosts

Method and system for peer-to-peer network latency measurement % http://www.google.com/patents/US6012096

