\newglossaryentry{metamodel}
{
	name={méta-modèle},
	description={un modèle de langage servant lui-même à la modélisation}	
}

\newglossaryentry{zeroconf}
{
	name={ZeroConf},
	description={un mécanisme normalisé de prise de connaissance des services offerts sur un réseau local, principalement connu grâce à l'implémentation d'\brand{Apple} nommée \brand{Bonjour}}	
}

\newglossaryentry{sensibilisee}
{
	first={\underline{sensibilisée}},
	name={sensibilisation (transition)},
	description={une transition est sensibilisée ssi pour toute place d'entrée, elle possède au moins autant de jetons que le poids de l'arc menant à la transition~\cite[p. 35]{allombert2009aspects}}	
}

\newglossaryentry{franchie}
{
	first={\underline{franchie}},
	name={franchissement (transition)},
	description={le franchissement d'une transition consiste en la suppression de jetons de chacune de ses places d'entrée et l'ajout de jetons dans chacune de ses places de sortie et ce proportionnellement à la pondération des arcs}	
}

\newglossaryentry{CMake}
{
	name={CMake}, 
	description={un outil de génération de chaîne de compilation multi-plateformes}
}

\newglossaryentry{contintegr}
{
	name={intégration continue},
	description={une méthode de développement logicielle qui compile et réalise des tests à chaque soumission de nouveau code sur le système de gestion de versions}
}

\newglossaryentry{macports}
{
	name=MacPorts, 
	description={un système de gestion de logiciels et de paquetages très utilisé sous \brand{Mac OS X}}
}

\newglossaryentry{graphviz}
{
	name=GraphViz, 
	description={un outil de manipulation de graphes permettant notamment la représentation graphique et le placement à l'aide de l'outil \brand{DOT}}
}
\newglossaryentry{modelica}
{
	name=Modelica, 
	description={un langage de programmation orienté objet basé sur des équations servant à modéliser des systèmes physiques}
}
\newglossaryentry{cyberphysique}
{
	name=cyber-physique, 
	description={la science des systèmes informatiques (souvent répartis) contrôlant des systèmes physiques et électro-techniques}
}

\newglossaryentry{calculambiant}
{
	name={calcul ambient}, 
	description={une \gls{algebreproc} orientée vers les périphériques mobiles et les réseaux à topologie dynamique}
}

\newglossaryentry{picalcul}
{
	name={$\pi$-calcul},
	description={un langage de programmation de conception proche du $\lambda$-calcul, servant à modéliser des systèmes répartis. C'est une \gls{algebreproc}}	
}

\newglossaryentry{joincalcul}
{
	name={join-calcul}, 
	description={un cas particulier du \gls{picalcul}, qui a été implémenté dans divers langages comme le \brand{Caml} ou le \brand{C++}}
}

\newglossaryentry{algebreproc}
{
	name={algèbre de processus},
	description={un ensemble de méthodes formalisées permettant de modéliser des systèmes répartis et concurrents}
}

\newglossaryentry{bigdata}
{
	name={Big Data},
	description={des ensembles de données extrêmement volumineux et difficiles à gérer par des méthodes standard, comme par exemple les résultats d'expériences scientifiques qui collectent sur une longue durée à court intervalle, ou bien les bases de données de sites comme \brand{Google}, \brand{Wikipédia}\dots}
}

\newglossaryentry{reactiveml}
{
	name={Reactive ML},
	description={Un langage de programmation dérivé de \brand{ML}, qui cible les systèmes interactifs, comme les jeux vidéos, ou les interfaces graphiques}	
}

\newglossaryentry{mutex}
{
	name=mutex,
	description={une méthode pour synchroniser des agents agissant sur les mêmes données, permettant d'assurer la correction de l'exécution d'un algorithme réparti}	
}

\newglossaryentry{logiquedehoare}
{
	name={logique de Hoare},
	description={un ensemble de règles de logique permettant de prouver la correction d'un logiciel}
}