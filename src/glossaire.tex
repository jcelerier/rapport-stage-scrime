\newglossaryentry{metamodel}
{
	name={méta-modèle},
	description={un modèle de langage servant lui-même à la modélisation}	
}

\newglossaryentry{graphviz}
{
	name=GraphViz, 
	description={un outil de manipulation de graphes permettant notamment la représentation graphique et le placement à l'aide de l'outil \brand{DOT}}
}
\newglossaryentry{modelica}
{
	name=Modelica, 
	description={un langage de programmation orienté objet basé sur des équations servant à modéliser des systèmes physiques}
}
\newglossaryentry{cyberphysique}
{
	name=cyber-physique, 
	description={la science des systèmes informatiques (souvent répartis) contrôlant des systèmes physiques et électro-techniques}
}

\newglossaryentry{calculambiant}
{
	name={calcul ambient}, 
	description={une \gls{algebreproc} orientée vers les périphériques mobiles et les réseaux à topologie dynamique}
}

\newglossaryentry{picalcul}
{
	name={$\pi$-calcul},
	description={un langage de programmation de conception proche du $\lambda$-calcul, servant à modéliser des systèmes répartis. C'est une \gls{algebreproc}}	
}

\newglossaryentry{joincalcul}
{
	name={join-calcul}, 
	description={un cas particulier du \gls{picalcul}, qui a été implémenté dans divers langages comme le \brand{Caml} ou le \brand{C++}}
}

\newglossaryentry{algebreproc}
{
	name={algèbre de processus},
	description={un ensemble de méthodes formalisées permettant de modéliser des systèmes répartis et concurrents}
}
