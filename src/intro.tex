\chapter*{Introduction}
Le domaine des systèmes répartis est vaste, et étudié depuis plusieurs décennies. Ce travail présente le cas d'une intersection entre ce domaine et l'informatique musicale, ou plus généralement la scénarisation interactive d'évènements. 

Ce stage s'est déroulé dans le cadre du projet ANR \ac{OSSIA}, en cours depuis (TODO). C'est un projet mêlant arts, science, et développement technologique (TODO citer rapport Thomas Meyssonier?) qui vise à concevoir et implémenter un cadre formel pour les scénarios interactifs, à l'attention des régies de spectacles, des compositeurs, ou encore (TODO, RSF? Blue Yeti?). 

Il y a donc deux étapes non entièrement disjointes qui sont étudiées dans le projet \ac{OSSIA} : la composition (création?), et l'exécution.

Un des modèles utilisé pour représenter ces scénarios est celui des réseaux de Petri, (TODO expliquer?), cependant, d'autres approches (TODO cf. Jaime) sont étudiées en parallèle. 

Le cœur de mon travail lors de ce stage a été d'étudier un moyen d'étendre l'exécution des réseaux de Petri à un réseau de machines, de manière à pouvoir étendre la puissance d'expression des outils scénaristiques proposés par \ac{OSSIA}.

Cela a aussi engendré la nécessité d'augmenter les capacités du logiciel lors de la composition, et l'introduction de nouveaux concepts dans le formalisme existant, tout en gardant la compatibilité avec ce qui avait déjà été fait.

Ce stage étant réalisé dans le cadre d'un double diplôme entre l'Université de Bordeaux et l'ENSEIRB-MATMECA, il comporte une partie orientée recherche, et une partie orientée ingénierie. Le présent document témoigne majoritairement du travail de recherche effectué; plus d'informations quant au processus de développement logiciel sont présentées dans le rapport pour l'ENSEIRB-MATMECA, normalement livré en annexe de celui-ci.

La problématique du sujet et l'environnement de travail seront d'abord présentés en détail, puis une revue de l'état de l'art sera effectuée. Le reste de ce document présente l'approche suivie pour résoudre le problème demandé, d'abord sur le plan théorique, puis au niveau de l'implémentation.