% Plus de figures
\renewcommand\floatpagefraction{.9}
\renewcommand\topfraction{.9}
\renewcommand\bottomfraction{.9}
\renewcommand\textfraction{.1}   
\setcounter{totalnumber}{50}
\setcounter{topnumber}{50}
\setcounter{bottomnumber}{50}

% C++ pour lstlistings
\lstset{
language=C++,
basicstyle=\footnotesize,
numbers=left,
numberstyle=\footnotesize,
stepnumber=1,
numbersep=5pt,
backgroundcolor=\color{white},
showspaces=false,
showstringspaces=false,
showtabs=false,
frame=single,
tabsize=2,
captionpos=b,
breaklines=true,
breakatwhitespace=false,
escapeinside={\%*}{*)}
}

% Accents dans commentaires lstlistings
\lstset{
  literate={ù}{{\`u}}1
           {é}{{\'e}}1
           {è}{{\'e}}1
           {à}{{\`a}}1
}

% Enelver espace au dessus de chapitre
\makeatletter
\def\ttl@mkchap@i#1#2#3#4#5#6#7{%
    \ttl@assign\@tempskipa#3\relax\beforetitleunit
    \vspace{\@tempskipa}
    \global\@afterindenttrue
    \ifcase#5 \global\@afterindentfalse\fi
    \ttl@assign\@tempskipb#4\relax\aftertitleunit
    \ttl@topmode{\@tempskipb}{%
        \ttl@select{#6}{#1}{#2}{#7}}%
    \ttl@finmarks  % Outside the box!
    \@ifundefined{ttlp@#6}{}{\ttlp@write{#6}}}
\makeatother

% Théorêmes usuels
\newtheorem{mydef}{Définition}
\theoremstyle{plain}
\newtheorem{myth}{Théorême}
\newtheorem*{mynot}{Notation}

% Commande \brand
\renewcommand*{\acsfont}[1]{\brand{#1}}
\newcommand{\brand}[1]{\textsc{\textbf{#1}}}

% Paragraphe avec saut de ligne
\let\oldpar\paragraph
\renewcommand{\paragraph}[1]{\oldpar{#1}\mbox{}\\}

% Opérateurs
\DeclareMathOperator{\entree}{E}
\DeclareMathOperator{\sortie}{O}
\DeclareMathOperator{\pre}{Pre}
\DeclareMathOperator{\post}{Post}
\DeclareMathOperator{\shared}{Shared}
\DeclareMathOperator{\memory}{Memory}