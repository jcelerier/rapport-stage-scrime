\documentclass[a4paper,11pt,oneside]{report}

\usepackage{setspace}
\usepackage[tmargin=2.5cm,bmargin=2.5cm,lmargin=3cm,rmargin=2cm]{geometry}
\usepackage[utf8]{inputenc}
\usepackage[french,english]{babel}
\usepackage[T1]{fontenc}
\usepackage{relsize}
\usepackage{color}
\usepackage{listings}
\usepackage{float}
\usepackage{kpfonts}
\usepackage{verbatimbox}
\usepackage{datetime}
\usepackage[version=0.96]{pgf}
\usepackage{tikz}
\usetikzlibrary{arrows,shapes,snakes,automata,backgrounds,petri}
\usepackage{amsthm}

% Pour figures en paysage
\usepackage{wrapfig}
\usepackage{lscape}
\usepackage{rotating}

\definecolor{dkgreen}{rgb}{0,0.6,0}
\definecolor{gray}{rgb}{0.5,0.5,0.5}
\definecolor{mauve}{rgb}{0.58,0,0.82}


% more figures per page
\renewcommand\floatpagefraction{.9}
\renewcommand\topfraction{.9}
\renewcommand\bottomfraction{.9}
\renewcommand\textfraction{.1}   
\setcounter{totalnumber}{50}
\setcounter{topnumber}{50}
\setcounter{bottomnumber}{50}

\usepackage{graphicx}
\usepackage{mdframed}
%% \newcommand{\foofoo}{\hspace{-2.3pt}$\bullet$ \hspace{5pt}}


\lstset{
language=C++,
basicstyle=\footnotesize,
numbers=left,
numberstyle=\footnotesize,
stepnumber=1,
numbersep=5pt,
backgroundcolor=\color{white},
showspaces=false,
showstringspaces=false,
showtabs=false,
frame=single,
tabsize=2,
captionpos=b,
breaklines=true,
breakatwhitespace=false,
escapeinside={\%*}{*)}
}


\lstset{
  literate={ù}{{\`u}}1
           {é}{{\'e}}1
           {è}{{\'e}}1
           {à}{{\`a}}1
}


\geometry{headheight=15pt}
\usepackage{fancyhdr}
\usepackage[xindy,toc]{glossaries}
\usepackage[numbers]{natbib}
\usepackage{fancyvrb}
\usepackage{float}
\usepackage{caption}
\usepackage{subcaption}
\usepackage{algorithm2e}
\usepackage{CJKutf8}
\usepackage{caption}
\usepackage[footnote,smaller]{acronym}
\usepackage[perpage]{footmisc}
\usepackage[]{titlesec}
\makeatletter
\def\ttl@mkchap@i#1#2#3#4#5#6#7{%
    \ttl@assign\@tempskipa#3\relax\beforetitleunit
    \vspace{\@tempskipa}
    \global\@afterindenttrue
    \ifcase#5 \global\@afterindentfalse\fi
    \ttl@assign\@tempskipb#4\relax\aftertitleunit
    \ttl@topmode{\@tempskipb}{%
        \ttl@select{#6}{#1}{#2}{#7}}%
    \ttl@finmarks  % Outside the box!
    \@ifundefined{ttlp@#6}{}{\ttlp@write{#6}}}
\makeatother

\makeglossaries

\newtheorem{mydef}{Définition}
\theoremstyle{plain}
\newtheorem{myth}{Théorême}

\pagestyle{fancy}

\renewcommand*{\acsfont}[1]{\brand{#1}}
\newcommand{\brand}[1]{\textsc{\textbf{#1}}}

\let\oldpar\paragraph
\renewcommand{\paragraph}[1]{\oldpar{#1}\mbox{}\\}

\DeclareMathOperator{\power}{power}
\DeclareMathOperator{\phase}{phase}
\DeclareMathOperator{\recompute}{recompute}

\DeclareMathOperator{\pre}{Pre}
\DeclareMathOperator{\post}{Post}
\DeclareMathOperator{\shared}{Shared}
\DeclareMathOperator{\memory}{Memory}


\acrodef{LaBRI}{Laboratoire Bordelais de Recherche en Informatique}
\acrodef{SCRIME}{Studio de Création et de Recherche en Musique Électro-acoustique}
\acrodef{OSSIA}{Open Scenario System for Interactive Application}
\acrodef{API}{Application Programmation Interface}
\acrodef{IPB}{Institut Polytechnique de Bordeaux}
\acrodef{OSC}{Open Sound Control}
\acrodef{NTP}{Network Time Protocol}
\acrodef{PTP}{Precision Time Protocol}
\acrodef{CRR}{Conservatoire à rayonnement régional}
\acrodef{MCLT}{Modulated Complex Lapped Transform}
\acrodef{LSB}{Least Significant Bit}
\acrodef{MSB}{Most Significant Bit}
\acrodef{RLSB}{Resistant LSB}
\acrodef{SSW}{Spread-Spectrum Watermarking}
\acrodef{FFT}{Fast Fourier Transform}
\acrodef{GUI}{Graphical User Interface}


\usepackage{hyperref}
\begin{document}
\begin{titlepage}
\vspace*{\stretch{2}} 
  \begin{center}

	\begin{spacing}{1.7}
    \textsc{\Huge Rapport de stage ~ \\ distribution des réseaux de Petri dans le cadre du logiciel i-score}\\[1cm]
    \textsc{\huge Travail réalisé dans le cadre du Master Recherche de l'Université de Bordeaux}\\[1cm]
    \end{spacing}
    \textsc{\Large Version de travail}
    
  \end{center}
  
  \vspace*{\stretch{2}}
  \begin{flushbottom}
   \begin{flushleft}
    Jean-Michaël \textsc{Celerier} - \today \\
   \end{flushleft}
  \end{flushbottom}
\end{titlepage}
\clearpage

\selectlanguage{english}
\begin{abstract}
TODO abstract (en anglais?)
\end{abstract}
\selectlanguage{french}

\tableofcontents
\listoffigures
\listoftables
\chapter*{Introduction}
\lettrine[lines=2]{L}{e} domaine des systèmes répartis est vaste, et étudié depuis plusieurs décennies. Ce travail présente le cas d'une intersection entre ce domaine et l'informatique musicale, ou plus généralement la scénarisation interactive d'évènements. 

Ce stage s'est déroulé dans le cadre du projet \brand{ANR} \ac{OSSIA}, en cours depuis 2012. Ce projet mêle arts, science, et développement logiciel, et vise à concevoir et implémenter un cadre formel pour les scénarios interactifs, à l'attention des régies de spectacles, des compositeurs, ou encore dans un cadre muséographique. 

Il y a donc deux étapes non entièrement disjointes qui sont étudiées dans le projet \ac{OSSIA} : la composition , et l'exécution.
Un des modèles utilisé pour représenter ces scénarios est celui des réseaux de Petri, cependant, d'autres approches, comme par exemple une formalisation à base de \gls{reactiveml} sont étudiées en parallèle. 

Le cœur de mon travail lors de ce stage a été d'étudier un moyen d'étendre l'exécution des réseaux de Petri à un réseau de machines, de manière à pouvoir étendre la puissance d'expression des outils scénaristiques proposés par \ac{OSSIA}.
Cela a aussi engendré la nécessité d'augmenter les capacités du logiciel lors de la composition, et l'introduction de nouveaux concepts dans le formalisme existant, tout en maintenant la compatibilité avec ce qui avait déjà été fait.

Ce stage étant réalisé dans le cadre d'un double diplôme entre l'Université de Bordeaux et l'ENSEIRB-MATMECA, il comporte une partie orientée recherche, et une partie orientée ingénierie. Le présent document témoigne majoritairement du travail d'ingénierie; plus d'informations quant au processus de recherche, ainsi qu'une bibliographie complète sont présentées dans le rapport délivré à l'Université de Bordeaux.

La problématique du sujet et l'environnement de travail seront d'abord présentés en détail. Puis l'approche suivie pour résoudre le problème demandé sera avancée, d'abord sur le plan théorique, puis au niveau de l'implémentation.
\chapter{Problématique et organisation}
\section{LaBRI et SCRIME}
Le stage s'est déroulé du deux février au treize juin, dans les bâtiments du \ac{SCRIME} et en tant que stagiaire du \ac{LaBRI}. L'équipe dont je fais partie est variable mais j'étais entouré de deux ingénieurs affectés à ce projet, ainsi que plusieurs autres stagiaires.

J'ai rencontré mes tuteurs, Mme. Desainte-Catherine et M. Chaumette, de manière très régulière, quasiment chaque semaine, pour faire un point sur mon avancement.

\section{Projet OSSIA}
Le stage s'inscrit dans le cadre du projet \brand{ANR} \ac{OSSIA}. Ce projet, démarré en 2012, apporte une réponse aux questions posées lors du projet \brand{ANR Virage}, qui avait pour but de dresser un état des lieux des besoins et pratiques en terme de contrôle et d'écriture dans le cas des contenus en temps réel, puis d'établir un cahier des charges pour une solution unificatrice pour la conception de scénarios, et l'écriture du temps dans un cadre intermédia.

Des acteurs d'origines diverses participent au projet \ac{OSSIA}.

\begin{itemize}
	\item Des laboratoires : \ac{LaBRI}, \ac{GMEA}
\end{itemize} 


(TODO historique).
\begin{mydef}
Scénario
\end{mydef}

\begin{mydef}
Processus
\end{mydef}

\subsection{Logiciel i-score}
Le logiciel i-score est en développement depuis plusieurs années, et comporte l'héritage de plusieurs autres logiciels et paradigmes qui ont été développé avant lui, comme visible en figure \ref{fig:heritageIScore}.

\begin{figure}[H]
	\centering
	\includegraphics[scale=0.4]{images/iscoreHeritage.png}
	\caption{Les racines du logiciel i-score}
	\label{fig:heritageIScore}
\end{figure}

Actuellement, il incarne à la fois l'outil permettant d'écrire les scénarios, et de les exécuter.

Deux versions d'i-score sont en développement parallèle : i-score 0.2, une version fonctionnelle mais qui n'inclut pas les travaux les plus récents du paradigme \ac{OSSIA}, et i-score 0.3, une version pour l'instant non-fonctionnelle, mais qui a pour vocation de représenter l'état actuel du développement théorique.
(TODO screenshots)

\subsection{Score et API OSSIA}
\subsubsection{Score}
Score est la librairie sous-jacente qui comporte la logique utilisée pour faire l'exécution d'i-score.

\subsubsection{API}
La version 0.3 d'i-score se base sur une \ac{API}, qui offre toutes les possibilités de gestion de scénario directement au programmeur, de manière à ce que d'autres applications puissent être conçues par dessus. Par exemple, des acteurs du jeu vidéo sont intéressés par le fonctionnement offert par \ac{OSSIA} dans un environnement \brand{Unity}. 

\section{Répartition et attentes}
\subsection{Problèmes posés par la répartition}
Un des problèmes majeurs ici est le besoin de synchronisation avec une horloge physique. En effet, les relations entre débuts et fins d'évènements sont données sous forme de millisecondes les séparant, les réseaux étant temporels. Ainsi, les mécanismes d'horloges vectorielles et logiques ne sont pas directement applicables.

Le deuxième problème est qu'il n'est pas réellement question de répartition au sens couramment entendu dans le domaine de l'informatique répartie. En effet, ici, les processus correspondent à des besoins physiquement présents sur des machines et qui ne peuvent pas être déplacés sur d'autres machines. Par exemple, une machine peut avoir une sortie vidéo, et une autre machine une carte son : cela conditionne pour le compositeur l'endroit ou s'exécute un processus, selon qu'il soit relatif à l'image ou au son.

Idéalement, il serait possible de faire une différenciation automatique, néanmoins, deux aspects s'y opposent : 
\begin{itemize}
\item Le compositeur désire avoir le choix pour la répartition.
\item Score ne communique qu'avec des messages \ac{OSC}. Un message \ac{OSC} a le format suivant : 
\begin{verbatim} /adresse/où/écrire argument1 argument2 ... \end{verbatim}.

Les adresses sont laissées à la discrétion des logiciels avec lesquels Score communique, car non standardisées. Il n'est donc pas possible de savoir quel type d'information est communiqué. 
\end{itemize}
Un troisième point, limitant, à tenir en compte, est celui de la gestion de l'interactivité. Dans ce cas, l'impact de la latence due au réseau physique est inévitable.
\section{Organisation au cours du stage}
(TODO mettre diagramme de Gantt)
\chapter{État de l'art}
\label{chapterStateOfTheArt}
L'état de l'art des domaines relatifs à ce stage s'intéressera à trois aspects : 
\begin{itemize}
	\item Les avancées en réseaux de Petri, qui sont au cœur du modèle.
	\item Les avancées en terme de répartition, et de synchronisation en temps entre machines.
	\item L'état actuel du projet \ac{OSSIA}. 
\end{itemize}

Quand ils ne sont pas semblables au français, les noms anglais des termes scientifiques employés seront marqués en \textit{italique}.

\section{Réseaux de Petri}
Les réseaux de Petri\cite{petri1962kommunikation} sont au cœur du formalisme utilisé ici. Néanmoins, une différence majeure entre leur utilisation dans le projet \ac{OSSIA} avec d'autres utilisations plus courante est que pour \ac{OSSIA}, les réseaux ne servent pas à faire de l'analyse statique mais sont au cœur du moteur d'exécution du programme. C'est-à-dire que la théorie des scénarios a été développée de manière à pouvoir être exprimée en terme de réseaux de Petri, et que ces réseaux de Petri sont ensuite exécutés avec un algorithme relativement simple, qui sera détaillé plus tard dans ce document.

La conférence principale sur le thème des réseaux de Petri est l'\textit{International Conference on Application and Theory of Petri Nets and Concurrency}, qui présente cette année sa 35\ieme édition.

\subsection{Familles de réseaux de Petri}
À l'origine, les réseaux portaient le nom de réseaux places-transitions. La définition la plus simple est celle d'un graphe orienté biparti, dont les deux parties sont les places, et les transitions.

Les places contiennent des jetons; quand les places antécédentes à une transition possèdent toutes au moins un jeton, celle-ci est dite \textit{sensibilisée}. Elle peut alors s'exécuter (on dit qu'elle est \textit{tirée}) en enlevant un jeton à chaque place la précédant et en rajoutant un jeton à chaque place la suivant.

Plusieurs cours et livres récents présentent les généralités des réseaux de Petri \citep[voir][]{david2010discrete, diaz2013petri}, avec les notations et vocabulaires contemporains.

Néanmoins, étant un formalisme facilement extensible, plusieurs modifications des réseaux de Petri ont prouvé leur intérêt; elles seront vues plus en détail dans les prochains paragraphes.
\subsubsection{Réseaux hiérarchiques}
L'idée de base des réseaux hiérarchiques consiste à avoir des sous-réseaux qui s'exécutent lorsqu'un jeton arrive dans une place.

Une extension aux transitions existe, en utilisant la notion de blocs de construction\cite{fehling1993concept}.

Actuellement, l'essentiel de la recherche dans ce domaine consiste en la recherche de formalisations pouvant généraliser les différents concepts de modularité, comme par exemple \brand{LLAMAS} (Language for Advanced Modular Algebraic Nets)\cite{colom2013application}.

Les réseaux hiérarchiques sont principalement utilisés pour modéliser des systèmes complexes du monde réel, dans des cas industriels par exemple.

\subsubsection{Réseaux colorés}
Les réseaux colorés\cite{zervos1977colored,jensen1987coloured} ont été une des premières extensions des réseaux de Petri. L'idée principale est de permettre aux jetons d'être porteurs de donnée, à l'aide des couleurs. 

Les transitions peuvent créer des jetons de couleur, et les arcs possèdent des fonctions d'arc laissant passer les couleurs choisies.

La notion de hiérarchie a été étendue aux réseaux colorés de plusieurs manières différentes \cite{rozenberg1991advances}, qui sont à choisir en fonction des cas d'utilisation.

Une généralisation des réseaux colorés porte le nom de réseaux de haut niveau (\textit{high-level Petri nets}) \cite{jensen1983high}. Elle remplace la notion de couleur par celle de type, plus extensible.
Il est possible de convertir des machines à état \brand{UML} en réseaux de Petri de haut niveau en utilisant un algorithme génétique \cite{alhroob2014transforming}.

\subsubsection{Réseaux liés au temps}
Il existe plusieurs manières d'introduire la notion de temps dans les réseaux de Petri.
\begin{itemize}
\item La première méthode a été celle des réseaux temporisés (\textit{timed Petri net}) : les places ou transitions portent les durées des actions qu'elles simulent. C'est le cas d'i-score.

\item Peu après a été établi le formalisme des réseaux temporels (\textit{time Petri net}) : les éléments du réseau (place ou transition) peuvent avoir une date minimale et maximale d'exécution.
Une extension récente \cite{klai2013temporal} au dessus des réseaux temporels  présente un moyen d'encoder le temps minimal et maximal écoulé à chaque instant, ainsi que les avantages que cela peut avoir.

\item Enfin, le modèle relatif au temps le plus récent utilise les opérateurs de logique temporelle \cite{logic2002temporal,suzuki1989temporal}. L'avantage par rapport aux modèles temporels et temporisés est une meilleure représentation des propriétés de vivacité (\textit{liveness}).
\end{itemize}
\paragraph{Réseaux stochastiques}
Les réseaux de Petri stochastiques \cite{bause1996stochastic} sont d'une certaine manière reliés au temps.

Dans ce type de réseau, les transitions franchissables sont activées après un délai déterminé par une variable aléatoire.
 
\subsubsection{Réseaux flous}
Les réseaux de Petri flous (\textit{fuzzy Petri nets}) \cite{pedrycz1994generalized} sont une classe de réseau de Petri inspirés des réseaux neuronaux. Leur utilité principale est comme pour les réseaux neuronaux l'apprentissage informatique.

Une approche récente utilise les réseaux de Petri flous pour modéliser des systèmes à connaissance floue\cite{wang2014dynamic}, et plus généralement les systèmes de modélisation de la connaissance, en prenant en compte la notion d'incertitude.

\subsection{Standardisation}
Il est assez apparent au vu de la section précédente que les variations autour du formalisme de base sont assez ample, et que beaucoup d'extensions sont possibles.

Un effort de standardisation est mené depuis le début des années 2000 pour avoir un format d'échange commun à différents outils, avec le langage \ac{PNML}.

Cet effort a abouti en 2008 et 2011 avec la standardisation dans la norme ISO/IEC 15909-2.

Le langage est basé sur du \brand{XML}, avec une attention importante à la modularité. Il a été conçu à l'aide de \glspl{metamodel} \brand{UML}.

Un état de l'art\cite{hillah2010standardisation} a été dressé par les auteurs du langage en 2010.

L'accent actuel est mis sur la gestion de la modularité, avec l'implémentation des concepts de hiérarchie, ou d'encapsulation. Ce travail est réalisé dans \brand{Modular PNML}. Un méta-modèle pour représenter les réseaux de Petri modulaires est en cours d'étude\cite{marechal2012modular}.

\subsection{Outils développés pour l'analyse des réseaux de Petri}
En sus de la théorie bâtie depuis bientôt un demi-siècle, les réseaux de Petri sont aussi très utilisé en pratique, pour réaliser des simulations de systèmes à complexité variable.

\subparagraph{ePNK} est une plateforme logicielle travaillant en format \ac{PNML}. Elle est bâtie autour de la notion d'extensibilité, en permettant aux autres développeurs d'insérer facilement leurs propres formalismes de réseaux de Petri avec une architecture en plug-in.

À la base prévue pour l'édition, elle est aussi désormais capable de simuler l'exécution de réseaux de Petri  \cite{kindler2013simulator}.

\subparagraph{PetriNet API} \cite{lohmann2009petri} est une interface de programmation en \brand{C++} offrant aux développeurs des structures de données simples (place, transition, réseau de Petri) permettant de bâtir leurs propres outils par dessus.

Un des avantages est l'import de données au format \ac{PNML} et l'export au format \brand{DOT} de \gls{graphviz}, à des fins de visualisation.

\subparagraph{CPN Tools} (\textit{Colored Petri Net Tools}) permet comme \brand{ePNK} d'éditer, simuler, et analyser des réseaux de Petri colorés \cite{jensen2007coloured}.

Il est notamment utilisé dans le projet \ac{OSSIA} à des fins de validation. Une capture d'écran est visible en \cref{fig.cpntools}.

\begin{figure}[H]
	\centering
	\includegraphics[scale=0.3]{images/cpntools.png}
	\caption{Capture d'écran : \brand{CPN Tools}}
	\label{fig.cpntools}
\end{figure}

\subparagraph{PNlib} \cite{pross2014object} est un outil de simulation orienté objet visant le cas particulier des processus biologiques.

La bibliothèque a été réalisé en \gls{modelica} et s'y intègre facilement.

\subparagraph{Snoopy} \cite{heiner2012snoopy} est un autre cadre logiciel à vocation unificatrice, permettant aussi l'utilisation de la couleur. Il supporte la hiérarchie.

Des outils d'analyse dynamique sont intégrés au logiciel.

Il a été appliqué aux réseaux biomoléculaires \cite{rohr2010snoopy}.

Une des particularités du logiciel est d'avoir plusieurs vues pour le même modèle, et d'aleterner facilement entre ces vues. Par exemple, il est possible d'avoir un modèle stochastique et un modèle continu pour le même réseau, qui partagent certaines propriétés.

\subsection{Applications récentes}
Il serait très dur d'établir un inventaire complet des utilisations des réseaux de Petri : comme pour les automates, les machines de Turing, ou encore le lambda-calcul, c'est un outil avec une très forte puissance d'expression, qui peut donc être utilisé dans une variété d'environnements différents, à des fins de modélisation.

Par exemple, à l'origine, C. A. Petri avait conçu les réseaux qui portent aujourd'hui son nom afin de modéliser des réactions chimiques. Ils sont maintenant très utilisés dans les modélisations de comportements biologiques \cite{koch2014petri}.

\subsubsection{Parallélisme}
Une des applications les plus courantes des réseaux de Petri est de simuler des systèmes répartis.

Un domaine annexe est celui de la parallélisation. Récemment, des progrès ont été fait dans la gestion de l'ordonnancement dans le cas d'applications s'exécutant en parallèle\cite{chen2014research}. Notamment, le système est modélisé à l'aide d'un réseau de Petri, et un algorithme génétique est utilisé pour y rechercher un chemin permettant un temps d'exécution minimal. 

\subsubsection{Application aux interfaces homme-système}
Les réseaux de Petri peuvent servir à modéliser toutes sortes de systèmes se basant sur une succession d'évènements.
 
Ils ont par exemple été utilisés pour la modélisation des interactions homme-système\cite{campos2014elementary}, avec différentes catégories d'évènements, comme le dépassement d'un seuil ou l'émission d'un signal, ce qui permet par la suite d'effectuer une analyse statique qui pourrait montrer des failles dans la conception du dit système.

\subsubsection{Application à des modèles cyber-physiques}
La \gls{cyberphysique} peut tirer un certain bénéfice des réseaux de Petri pour la simulation de son bon fonctionnement. En effet, elle consiste en différents éléments répartis et communiquants.

Des réseaux de Petri spatio-temporels ont été introduits\cite{zhang2014modeling} pour gérer les aspects de déplacements physique au cours du temps dans ce cas.

\subsubsection{Application aux jeux vidéos}
Les jeux vidéos peuvent parfois se présenter comme des processus scénarisés, que ce soit dans le temps, ou dans l'espace.
Une autre famille de réseaux spatio-temporels ont été introduits dans \cite{natkin2004new}, servant à prévoir et agir en fonction des déplacements des joueurs dans des niveaux de jeu.

Plus récemment, une classe de réseaux de Petri dédiée à des buts organisationnels, les \textit{WorkFlow nets}\cite{oliveira2011game}, ont été appliqués à la modélisation de jeu vidéo.
\section{Répartition}
La répartition (\textit{distribution}) est un second point important de ce stage. C'est un domaine qui est souvent très proche de la technique : le besoin d'utiliser des algorithmes répartis provient souvent de contraintes matérielles ou temporelles.

\subsection{Ouvrages de référence}
Plusieurs livres sont dédiés à des présentations générales du sujet, celui-ci étant étudié depuis plusieurs décénnies.

Au niveau des algorithmes, le livre de Nancy Lynch \cite{lynch1996distributed} présente les cas et problèmes les plus courants, en séparant les méthodes synchrones et asynchrones. 

Un autre ouvrage \cite{attiya2004distributed} met plus l'accent sur la simulation d'algorithmes distribués.

\subsection{Vérification}
La vérification des algorithmes distribués est une branche majeure qui nécessite des outils particuliers par rapport aux méthodes formelles usuelles, comme la \gls{logiquedehoare} telle qu'elle fut définie à l'origine par exemple.

Les outils de base sont la sécurité (\textit{safety}) et la vivacité (\textit{liveness}), telles qu'énoncées par Leslie Lamport \cite{lamport1977proving}.

De nombreux articles plus ou moins récents sur ce sujet sont référencés par un laboratoire de l'université de l'Illinois \cite{formaldistributed}.

Un des aspects particulièrement important dans le cas d'algorithme distribué est celui de la tolérance aux pannes. Par exemple, comment faire si un câble est débranché ? Et, plus précisément ici, comment vérifier qu'un algorithme donné sera tolérant aux pannes possibles ? 

Un exemple est donné \cite{mcmillan2014verification} pour le cas des algorithmes de consensus, qui introduit une logique de premier ordre particulière (nommée \textit{Consensus verification logic}).

Une autre approche utilisée est celle des algorithmes de transmission de messages \cite{jezequel2014message}, qui a été appliquée à la vérification de certains protocoles distribués, comme par exemple un protocole de \gls{mutex}.

\subsection{Développement}
Cette partie traitera des publications récentes ayant trait à l'implémentation des algorithmes de répartition.

Un ouvrage récent \cite{varela2013programming} fait le lien entre la théorie et la pratique, en présentant les méthodes les plus récentes pour écrire certains algorithmes. Un accent est mis sur la manière dont différents langages de programmation vont permettre plus ou moins simplement de réaliser différents paradigmes. Il tient aussi compte des avancées sur le \gls{calculambiant}, le \gls{picalcul}, et le \gls{joincalcul}.

\subsubsection{Modèles au fait des données (\textit{data-aware})}
Un algorithme opère toujours sur des données, néanmoins les opérations relatives à ces dernières (déplacement, envoi...) sont souvent considérées comme secondaires et comme quelque chose sur lesquelles l'auteur de l'algorithme distribué n'a pas de pouvoir.

Néanmoins, avec l'avènement du \gls{bigdata}, il est nécessaire de mieux prendre en compte cet aspect, parfois au niveau du système d'exploitation, ou encore des protocoles de routage utilisés \cite{baehni2004dependable} car les effets de bord deviennent trop volumineux.

Ainsi, des nouveaux paradigmes d'ordonnancement ont été proposés \cite{yildirim2012data, kosar2009paradigm}, ainsi que des méthodes pour optimiser le transfert de ces données.

\subsubsection{Routage}
Le routage est une problématique centrale des systèmes répartis, l'idéal étant de minimiser son temps.

Une méthode a été récemment proposée pour réduire ce temps en cas de forte charge sur les lignes, mais uniquement dans des cas de réseaux statiques \cite{jeon2014fully}.

% % % % Modularity in the design of robust distributed algorithms

De plus, il est intéressant de noter que des problèmes de routage comme la gestions des congestions peuvent s'appliquer à d'autres types de réseaux que les réseaux informatiques. Par exemple, des algorithmes ont été développés pour gérer en temps réel les problèmes des feux de circulation \cite{aoxue2014distributed}, en s'inspirant du concept de \gls{mutex} et en utilisant des capteurs embarqués sur chaque véhicule.

\subsubsection{Réseaux sans-fil}
Les réseaux sans-fil sont désormais omniprésents, et offrent de nouvelles opportunités d'optimisation qui leur sont propres.

Par exemple, il a été montré dans \cite{hosseinabadi2014exploiting} que les paquets entendus par un client mais qui ne lui étaient pas directement destinés (\textit{overheard packets}) pouvaient être utilisés pour améliorer la performance du réseau (et notamment les temps d'attentes dûs aux collisions), à l'aide d'algorithmes spécifiques.

Un outil, \brand{FlockLab} \cite{lim2013flocklab} a été conçu spécifiquement pour la vérification et l'analyse de réseaux sans-fil embarqués. Il permet le débogage, et l'étude des timings à une précision proche de la microseconde en utilisant les protocoles généralement utilisés sur l'embarqué, comme \ac{GPIO}.  

\section{Synchronisation}
Les problèmes de synchronisation sont généralement vus comme une sous-partie des problèmes inhérents aux systèmes embarqués. Néanmoins, dans ce sujet, c'est une des partie les plus importantes et pertinentes, d'où l'accent qui y est apporté.

La synchronisation implique toujours un concept d'horloges. Deux branches sont étudiées : les horloges logiques, qui servent à ordonner les éléments entre eux, et les horloges physiques, qui servent de référence absolue.

\subsection{Synchronisation logique}
\subsubsection{Origines : les horloges de Lamport}
Le premier système d'horloge logique est celui conçu par Leslie Lamport, récipient du prestigieux \textit{Turing Award 2013}.

Il a notamment conçu les horloges et horodatages qui portent son nom, et qui expriment la notion de "arrivé-avant".
\subsubsection{Horloges vectorielles}
Les horloges vectorielles (\textit{vector clocks}) sont une amélioration au dessus du concept des horloges de Lamport.

Néanmoins, plutôt que de ne faire garder que leur état logique aux nœuds, on leur fait garder l'état de tous les nœuds dont ils ont reçu des messages dans un vecteur. Lorsqu'un nœud envoie un message à un autre, ce dernier se met à jour avec les horodatages maximaux contenus dans le vecteur qui est envoyé.

Le problème des horloges vectorielles est la croissance de la taille du vecteur dans un système contenant beaucoup de nœuds.

\paragraph{Horloges plausibles}
Une méthode pour régler le problème de taille de vecteur a été proposée dans \cite{torres1999plausible}. Cela améliore l'efficacité dans le cas de grands réseaux, néanmoins il arrive que des évènements pouvant se dérouler en parallèle soient réorganisés pour se dérouler séquentiellement.

\subsection{Synchronisation physique}
Comme i-score est un logiciel permettant de manipuler le temps, il nécessite de travailler avec une horloge physique. Le problème principal des systèmes distribués se basant sur une horloge est de s'assurer que tous les nœuds ont la même heure.

\subsubsection{Algorithmes couramment utilisés}
En pratique, il existe deux méthodes couramment utilisées pour garantir une synchronisation des horloges entre machines.

\subparagraph{NTP} (\textit{Network Time Protocol}) est la méthode couramment utilisée sur Internet pour synchroniser l'heure des machines.

Elle emploie une structure en arbres avec plusieurs niveaux, chacun communiquant uniquement avec ses niveaux supérieurs et inférieurs.

Elle permet une synchronisation de l'ordre de quelques millisecondes seulement à l'échelle de l'internet \cite{mills1991internet}.
\subparagraph{PTP} (\textit{Precision Time Protocol}) est une méthode qui a été développée après \brand{NTP}, et qui est optimisée pour les réseaux locaux requérant une plus grande précision que celle que peut offrir \brand{PTP}.

Elle permet une synchronisation de l'ordre de $\num{1}\si{\micro\second}$ \cite{peng2009research, scheiterer2009synchronization}. La viabilité des algorithmes utilisés par \ac{PTP} sur systèmes \brand{Android} a été démontrée dans \cite{hsu2012measurement}.

En revanche, le nombre de paquets transmis est plus élevé que dans le cas de \ac{NTP}, ce qi peut poser problème en cas de charge élevée sur le réseau.
\subsubsection{Autres approches}
Une méthode dédiée aux réseaux ayant une possibilité de \textit{broadcast}, c'est-à-dire de transmission à tous les clients, a été présentée dans \cite{elson2002fine}. Elle offre de meilleures performances que \ac{NTP}, et possède comme différence avec les méthodes couramment utilisées la non-transmission de timestamps : les horloges sont déduites à partir des temps d'arrivée des messages.

D'autres approches ont essayé d'utiliser d'autres couches que les couches applicatives pour synchroniser les horloges. C'est le cas de \cite{abari2014one} : une approche dédiée aux réseaux sans-fil, qui modifie directement les couches physiques pour qu'elles se synchronisent entre elles automatiquement sur une horloge unique, présente dans le signal sans-fil, plutôt que sur leurs horloges internes généralement gérées par des quartz dont la fréquence peut être légèrement variable.

Enfin, des recherches ont été faites pour gérer les cas de réseaux peu fiables et possédant de grandes latences. Un mécanisme pour gérer les synchronisations par transmission acoustique dans des milieux marins est présenté dans \cite{syed2006time}. Ces cas ont des contraintes très importantes en terme de débits et de pertes de bits, ce qui force l'utilisation d'algorithmes dédiés, qui prennent notamment en compte le décalage d'horloges (\textit{clock skew}) qui peut être particulièrement important.
\subsection{Estimation de la latence}
\label{section:latence}
Il est critique dans de nombreuses applications collaboratives et réparties de pouvoir estimer la latence entre les différentes machines, et ce de la manière la plus précise possible. Cela peut s'appliquer dans le cas de réseaux locaux à haute vitesse, ou bien de l'internet.

L'algorithme le plus simple consiste en un message de type ping/pong entre deux machines, mais la précision est très faible, et il n'est pas possible de connaître la latence dans une seule direction.

Des méthodes plus perfectionnées peuvent par exemple faire appel à une machine tierce et des points de repères présents sur le réseau : \cite{banerjee2012network}, ou encore à des comportements entièrement pair-à-pair \cite{im2000method} qui fonctionnent sur des versions avancées de l'algorithme naïf présenté au paragraphe précédent. 
Une autre approche, purement logicielle, nommée \brand{King} \cite{gummadi2002king}, fait appel à des requêtes \ac{DNS} récursives. Ainsi, cette méthode se destine principalement à une utilisation sur internet.

\section{Projet OSSIA et partitions interactives}
Le projet \ac{OSSIA} utilise les acquis du projet \brand{Virage} \cite{baltazar2009virage} qui l'a précédé, ainsi que ceux de différents formalismes de partitions interactives. Il est aussi en interaction avec le projet \brand{INÉDIT} \cite{fober2013caracterisation} qui vise à rendre interopérable différents outils de création musicale et de composition.

\subsection{Partitions interactives}
\subsubsection{Origines : Boxes}
Les premières recherches ont porté sur une méthode de combiner des structures fréquentielles entre elles afin de composer un morceau. Ces travaux, réalisés aux alentours de l'an $\num{2000}$ ont posé les bases des développements actuels, en introduisant les concepts de contraintes, de boites temporelles et de hiérarchisation dans le même logiciel \cite{beurive2000logiciel}, nommé \brand{Boxes}.
\subsubsection{Premières méthodes}
Les partitions interactives ont été à l'origine conçues pour offrir un formalisme, et donc des possibilités de provabilité et de vérification, au cas de la composition musicale. 
L'idée originale a été de concevoir les partitions interactives comme un ensemble d'objets temporels rattachés à des relations d'Allen \cite{allen1983maintaining}, qui sont une forme de contraintes logiques adaptées aux évènements temporels. Les objets temporels peuvent contenir un processus et une contrainte locale \cite{allombert2007system}, et peuvent faire partie de différentes classes prédéfinies afin d'aider le compositeur. 

L'interactivité provient du fait que des évènements peuvent être déclenchés en temps-réel, sur scène, par un compositeur ou musicien, tout en maintenant une logique qui aura été écrite au préalable. Par exemple, la lecture d'une vidéo $V$ suit automatiquement la lecture d'un son $S$, lui-même déclenché lors de l'activation d'un capteur extérieur. 

Une des problématiques principales est donc de maintenir les contraintes entre objets temporels, tout en permettant l'interactivité.

Plusieurs méthode ont été étudiées pour répondre à ce problème :
\begin{itemize}
\item Une méthode par contraintes concurrentes, inspirée de \ac{NTCC} \cite{allombert2006concurrent}.
\item Une méthode par réseaux de Petri \cite{allombert2007system, allombert2009aspects}.
\end{itemize}

Ces méthodes ont été implémentées, avec de multiples extensions (présentées en détail dans \cite{mauricio2012structured} en section 3.2) dans les différents logiciels présentés en \cref{figheritageIScore}.

\subsection{Précision des formalismes}
Un des problèmes dans les méthodes présentées par Antoine Allombert est l'absence de modèle réunissant branchements et relations temporelles.

Une solution possible, nommée \textit{Time Conditional-Branching Scores}, est avancée par Mauricio Toro dans sa thèse \cite{mauricio2012structured}.

\subsection{Problématique du traitement de signal}
Bien qu'à l'origine, le lien entre partitions interactives et traitement du signal était fort, notamment dans le logiciel \brand{Boxes}, les recherches plus récentes ont porté majoritairement sur un formalisme uniquement scénaristique, sans réelle préoccupation de ce qui était scénarisé.

Un des problèmes principaux rencontrés est la gestion du temps : en effet, les cartes sons ont leur propre fréquence d'horloge, qui est généralement de $\num{44100}\si{\hertz}$, et fonctionnent par buffers, ce qui n'est pas toujours le cas des logiciels implémentant les différents formalismes de partitions interactives.

Néanmoins, des approches ont été menées en parallèle pour essayer de relier les deux. Mauricio Toro présente dans sa thèse une méthode pour inclure des processus \ac{FAUST} dans une partition interactive \cite[chapitre 8]{mauricio2012structured}.
Cette approche est continuée par Florent Berthaut avec le travail mené sur \brand{libLAScore}, une bibliothèque intégrant \brand{LibAudiostream} (qui implémente \ac{FAUST}), et \brand{libIscore} \cite{desaintecatherine2014interactive}.

Actuellement, un autre modèle entièrement basé sur la sémantique des réseaux de Petri, qui fonctionne par envoi de jetons correspondant aux paquets de donnée, est en cours de développement par Jaime Arias \cite{arias2014modelling}.

\subsection{Développement dans OSSIA}
Le projet \ac{OSSIA} se place à un niveau plus abstrait que les différents formalismes de partitions interactives.


\chapter{Apports au problème}
\label{chapterApports}
\section{Introduction}
Le but de cette partie est de fournir une sémantique  personnalisée pour la distribution de l'exécution d'un réseau de Petri sur un réseau informatique, en prenant en compte les délais dûs à la répartition physique des machines et les contraintes issues du projet \ac{OSSIA}.

\subsection{Rappel du formalisme}
Le formalisme de base utilisé est celui des réseaux de Petri t-temporisés, disposant d'un couple place $\rightarrow$ transition initiale et transition $\rightarrow$ place finale, avec pour restriction le fait qu'une place ne puisse posséder qu'une entrée et qu'une sortie, et ce pour éviter les problèmes de conflits et confusions : cela en fait des graphes d'évènements\cite[p. 7--9]{david2010discrete}. L'utilisation de la couleur est possible.

Les réseaux de Petri utilisés ici sont connexes : la seule place n'étant pas en sortie de transition est la place initiale, et la seule place n'étant pas en entrée de transition est la place finale.

\subsubsection{Algorithme d'exécution de base}
(TODO rédiger avec algorithm2e)
Un jeton est placé dans la place initiale, le temps est initialisé à 0. Une horloge est initialisée.

Pour chaque transition, si elle est sensibilisée, et que la somme des durées entre elle et la place initiale est écoulée, alors elle est franchie.

Quand tous les jetons ont franchi la transition menant à la place finale, l'algorithme termine.

\paragraph{Note} Il est possible que l'algorithme ne termine pas : en effet, on pourrait vouloir un scénario en boucle infinie, pour une installation artistique par exemple.

\subsection{Démarche suivie}
La première étape de mon travail a été de rechercher l'existant proche du cas présent. Néanmoins, malgré diverses rencontres avec des professeurs spécialisés dans le domaine des automates et des réseaux, et une recherche bibliographique sur ce sujet, aucun autre cas de moteur d'exécution de réseau de Petri n'a été trouvé.

Ainsi, ma recherche s'est plutôt portée sur les moyens de répartir des algorithmes à la base prévus pour une utilisation sur une seule machine. 

L'idée principale a été de lancer les exécutions des sous-scénarios répartis sur d'autres machines en avance.

Cela engendre deux problèmes : 
\begin{itemize}
\item Comment estimer le retard entre deux machines ?
\item Comment modifier les réseaux de Petri de manière à ce qu'ils prennent en compte le départ en avance qui en découle ?
\end{itemize}

C'est surtout le second qui est au cœur de cette partie, le premier étant couvert de manière extensive par la littérature scientifique.

Trois idées différentes pour parvenir à ce but sont présentées : deux qui se basent sur la modification de la topologie du réseau, et une qui se base sur la modification de la sémantique du réseau.

De plus, la notion de segment est introduite, qui est une légère abstraction permettant de représenter simplement les processus dans les réseaux de Petri.

\section{Définitions}
Un des problèmes de vocabulaire principal dans ce document est que nous faisons référence à deux types de réseaux tout à fait distincts : les réseaux de Petri, et les réseaux informatiques présents dans le monde physique.

Pour pallier ce problème, nous utiliserons de préférence l'idée de graphe et de nœuds quand il s'agit de réseau informatique, et l'idée de réseau quand il s'agit de réseau de Petri.

\subsection{Définitions relatives aux réseaux physiques}
\begin{mydef}[Réseau informatique]
Un réseau informatique est représenté par un graphe orienté $G$ dont les nœuds représentent les machines, et les arêtes les liens entre ces machines. 

Les arêtes sont étiquetées par un temps $\delta$, signifiant le temps de transmission d'un message entre les deux nœuds attenants à l'arête.
\end{mydef}
$\delta$ sera considéré fixe. Néanmoins, dans un scénario réel, il serait envisageable (et recommandé) de mesurer et corriger $\delta$ régulièrement en raison de la latence entre les nœuds qui peut varier, notamment à l'aide des méthodes présentées en partie \ref{section:latence}.
\\
\begin{mynot}
	Soient $N_1$ et $N_2$ deux nœuds du graphe. On note $\delta_{N_1 \rightarrow N_2}$ le temps de transmission entre $N_1$ et $N_2$.
\end{mynot} 

Bien que dans de nombreux cas, $\delta_{N_1 \rightarrow N_2} \simeq \delta_{N_2 \rightarrow N_1}$, nous ne ferons pas cette supposition ici. En revanche, nous considérerons toujours des liens en duplex.

Les nœuds de $G$ peuvent contenir des places et des transitions. Si un nœud contient une transition, on dit qu'il l'\textit{exécute}.

\subsection{Définitions relatives aux réseaux de Petri}
\begin{mydef}[Segment]
Un segment est composé d'une transition, de ses $N$ places d'entrées, et de ses $M$ places de sortie, avec $N \geq 1, M \geq 1$.
\end{mydef}

On notera que mis à part les segments initiaux et finaux, tous les segments partagent au moins une place avec un autre segment.

L'approche utilisée dans les schémas de ce document est de représenter un segment par ses places d'entrée et sa transition.

\begin{mynot}
On note $\entree(s)$ l'ensemble des places d'entrée de $s$, $\sortie(s)$ l'ensemble des places de sortie de $s$.
\end{mynot}

\begin{mynot}
On note $\pre(s)$ l'ensemble des segments précédant le segment $s$, $\post(s)$ l'ensemble des segments suivant $s$.
\end{mynot}

\begin{mynot}
On note $\shared(s_1, s_2)$ les places en sortie de $s_1$ et en entrée de $s_2$.
\end{mynot}

\section{Méthodes par modification de la topologie}
\chaptermark{Premières méthodes}
Ces formalisations se font par modification de la topologie du réseau de Petri. C'est-à-dire que des places, arcs, et transitions vont être rajoutées au réseau pour encoder la notion de départ en avance.

\subsection{Définitions}
\subsubsection{Définition des opérations de transformation de base}
Deux niveaux d'opération sont décrits dans ce document.

Ici, on décrit les opérations atomiques pouvant être exécutées sur les segments, qui ne tiennent pas compte de la répartition. 

Par la suite, chaque méthode comportera ses propres opérations, qui serviront spécifiquement à appliquer la répartition. 

Toutes ces opérations seront écrites en police informatique : \texttt{maFonction(x)}.

Il convient de noter que les opérations appliquant la répartition sont réversibles. Ainsi, il est toujours possible d'appliquer les opérations atomiques sans avoir besoin de tenir compte de la répartition : il suffit d'inverser ces dernières, d'appliquer les opérations atomiques, puis de ré-appliquer les opérations de répartition effectuées, si possible (un cas impossible serait celui ou on voudrait supprimer un segment qui était réparti).

\paragraph{Création}
On considère la fonction \texttt{crée(d, nPlacesEntrée, nPlacesSortie)}. Elle permet de créer un segment, de durée \texttt{d} et avec un nombre explicite de places d'entrée et de sortie.

\begin{itemize}
\item \texttt{d} peut être une durée variable.
\item \texttt{nPlacesEntrée} $ \geq 1 $ et \texttt{nPlacesSortie} $ \geq 1 $.
\end{itemize}
\paragraph{Lien}
Le lien crée un arc entre deux segments.

On définit : \texttt{lie(sOrigine, pDestination)} qui crée un arc entre la transition du segment \texttt{sOrigine}, et la place \texttt{pDestination}.

De même, \texttt{délie(sOrigine, pDestination)} supprime un arc.

\paragraph{Suppression}
(TODO processus sur place partagée?)
\texttt{suppr(s)} supprime le segment, ainsi que les arcs en entrée et en sortie.

Si un segment ne se retrouve plus relié à rien suite à cette opération, une place d'entrée lui est rajoutée, qui est reliée en sortie de la transition initiale.

\subsubsection{Définition de segments particuliers}
(TODO cela aurait-il sa place dans la partie plus générale ? Moins utilisé dans formalisation par couleur...)

Ces segments relient les notions de graphe de machines et de réseau de Petri.
\begin{mydef}[nd-transition]
Une transition non-distribuée, notée nd-transition, correspond à une transition dont aucune des places d'entrée ou de sortie ne sont situées sur un autre nœud du graphe.
\end{mydef}

\begin{mydef}[d-transition]
Une transition distribuée, notée d-transition, correspond à une transition dont une des places d'entrée ou de sortie est située sur un autre nœud du graphe.
\end{mydef}

\begin{mydef}[nd-segment]
Un d-segment (resp. nd-segment) est un segment bâti autour d'une d-transition (resp. nd-transition).
\end{mydef}

\begin{mydef}[Segments consécutifs]
Deux segments sont dits consécutifs s'ils partagent au moins une place. Cela implique qu'au moins une place de sortie de l'un soit une place d'entrée de l'autre.
\end{mydef}

Un réseau de Petri distribué contient des d-segments et des nd-segments consécutifs, ainsi qu'une place initiale et une place finale.

Des processus existent de plus au-dessus de segments, qui sont signifiés par un changement d'état d'un paramètre entre deux places. Il est donc important d'assurer que la durée de ces processus ne change pas lors de la distribution. (TODO déplacer dans pbq)

\subsection{Première tentative de formalisation}
Cette section présente la première idée qui a été pensée pour résoudre ce problème.

L'idée principale est d'exécuter les segments en les coupant au moment ou les segments suivants sont sensés partir en considérant les délais de transmission entre nœuds, puis de continuer leur exécution dans un autre segment.
 
\subsubsection{Opérations définies sur segment}
On considère que dans l'état initial du système, tous les segments sont situés sur le même nœud.

\paragraph{Déplacement}
L'opération de déplacement consiste à déporter l'exécution d'un nd-segment d'un nœud vers un autre, en transformant le réseau pour s'accommoder des délais.

\subparagraph{Processus} ~ \\

On note $s$ le segment à déplacer.
Pour tout segment $p \in \pre(s)$ : 

\begin{enumerate}
\item Soit $t_p$ la durée de la transition de $p$. On change cette durée en $t_p - \delta$.
\item On rajoute une place en sortie de $p$ ainsi qu'une transition $T$ de durée $\delta$ à la suite de cette place. 

Pour tout segment $o$ appartenant à $\post(p)$ à l'exception de $s$, on supprime les arcs entre $o$ et $p$, puis on crée des arcs allant de $T$ aux places d'entrée de $o$.

On duplique de plus toute place appartenant à $\shared(p, s)$, en sortie de $T$, de manière à garder l'ancienne position de $s$ en mémoire pour annuler le déplacement si nécessaire.

On notes cet ensemble de places dupliquées $\memory(s, p)$ (la mémoire des places d'origine lors du déplacement de $s$ par rapport à $p$).
 
\item On déplace l'information de fin de processus potentiellement contenue sur les places de $\shared(p, s)$ vers les places  $\memory(s, p)$. Cela permet d'avoir les évènements de fin de processus attenants à $p$ exécutés sur le même nœud que $p$.
\end{enumerate}

Puis, on applique la même opération entre $s$ et les segments appartenant à $\post(s)$, pour avoir une compensation de délai par rapport aux nœuds des segments suivant $s$.

Un exemple graphique est présenté en figure \ref{fig:deplacementMethode1}.

\begin{figure}[h!]
\centering
\begin{tikzpicture}[node distance=1.3cm,>=stealth,bend angle=45,auto]

  \tikzstyle{segment}=[]
  \tikzstyle{place}=[circle,thick,draw=blue!75,fill=blue!20,minimum size=6mm]
  \tikzstyle{red place}=[place,draw=red!75,fill=red!20]
  \tikzstyle{transition}=[rectangle,thick,draw=black!75,
  			  fill=black!20,minimum size=4mm]
  			  
  \tikzstyle{black box}=[draw=black, fill=black!30, draw opacity=0.7, fill opacity=0.2, thick, dashed, inner sep=0.5cm]


  \begin{scope}
    \node [transition] (Tk) [label=$t_p$] {};
	\node [segment] (Pk) [left of=Tk] {$\pre(p)$};
	\node [segment] (S) [right of=Tk] {$s$};
	\node [segment] (Sk) [below of=S] {$\post(p) \setminus s$};
			
  	\path
  		(Pk) edge[post] node {} (Tk)
  		(Tk) edge[post] node {} (S)
  			 edge[post] node {} (Sk);
  
   \begin{pgfonlayer}{background}
    \node [black box, fit=(Tk) (Pk) (S) (Sk)] (Box1) {};
    \node[above right] at (Box1.south west) {Nœud $N$};
   \end{pgfonlayer}
  \end{scope}
\end{tikzpicture}

  
\caption{Exemple : déplacement, première méthode}
\label{fig:deplacementMethode1}
\end{figure}

\paragraph{Recombinaison}
L'opération de recombinaison intervient quand un segment est déplacé sur un nœud où un autre segment avec lequel il partage une place est présent.
La recombinaison consiste en la fusion entre les places de $\shared(p, s)$ et de  $\memory(s, p)$ de telle sorte que le temps de démarrage du second segment soit inchangé.

La recombinaison est l'opération inverse du déplacement.

\subsubsection{Problèmes}
Il y a un problème majeur avec cette approche : la soustraction qui est effectuée à la durée de la transition de $p$. En effet, si on désire effectuer plusieurs opérations de déplacement successives, on peut arriver rapidement à une durée de transition négative, ce qui n'est pas acceptable. 

Cela est dû au fait que les transitions de $\pre(s)$ sont brisées en deux (ou plus) sous-transitions, de durées inférieures.

Il faut donc chercher une méthode qui permette plus de souplesse, et qui préserve les nd-segments dans leur forme originale.

\subsection{Seconde tentative de formalisation}
Cette formalisation est similaire à la précédente, car elle change aussi la topologie. Néanmoins, elle permet une plus grande flexibilité et ne pose pas de problèmes en cas d'enchaînements de déplacements.
Elle est cependant plus complexe à mettre en œuvre.

L'idée est la suivante : nous allons introduire deux segments spéciaux, ayant chacun un but distinct.

Le premier servira à dupliquer l'exécution d'un segment en en créant plusieurs branches pouvant s'exécuter en parallèle, en étant placées en précédence d'un segment réel (contenant un processus) ou fictif.

Le second servira à ajuster la durée des processus fictifs, ce qui permet d'ajuster l'instant auquel démarre le segment à déplacer par rapport aux contraintes du réseau.

Enfin, un troisième segment sera introduit plus tard, en partie \ref{section:synchroPetri}, permettant de synchroniser la fin des branches.

\subsubsection{Segments spéciaux}
\label{section:alphasegment}
\begin{mydef}[$\alpha$-segment]
Un $\alpha$-segment est un nd-segment particulier de durée $0$, possédant au moins deux places de sortie et une place d'entrée.
\end{mydef}
Un $\alpha$-segment peut être créé en précédence d'un segment $s$ de la manière suivante : 
\begin{enumerate}
\item Une transition $T_\alpha$ de durée 0 est créée. Toutes les places d'entrée de $s$ sont déconnectées de $s$ et placées en entrée de $T_\alpha$.
\item Une place est créée en sortie de $T_\alpha$ et en entrée de $s$. 
\end{enumerate}

Le rôle des $\alpha$-segments est de permettre à un segment d'être suivi par un nombre quelconque d'autres segments s'exécutant sur d'autres nœuds, avec des délais différents entre les nœuds.

Dans les schémas qui suivent, il sera représenté en \textcolor{BrickRed}{rouge}.

\begin{mydef}[$\beta$-segment]
Un $\beta$-segment est un d-segment particulier dont la durée vaut celle d'un autre segment moins un délai variable. Il précède nécessairement un nd-segment, et ne peut en être dissocié.
\end{mydef}
Son rôle est de simuler l'exécution d'un autre segment en ajustant sa durée de manière à ce qu'elle prenne en compte le lien entre deux nœuds, lorsque deux segments consécutifs sont placés sur deux nœuds différents.

Dans les schémas qui suivent, il sera représenté en \textcolor{OliveGreen}{vert}.
\subsubsection{Opérations définies sur segments}
\paragraph{Déplacement et recombinaison}

Il convient de noter que l'opération de déplacement ne peut produire, dans l'idéal, de résultats identiques aux cas non répartis que : 
\begin{itemize}
\item Si le délai entre les deux nœuds est inférieur à la durée des segments précédents.
\item Si la durée des segments précédents est prévisible (fixe).
\end{itemize} 

\subparagraph{Processus}
Soit $s$ le segment à déplacer, sur un nœud $N_d$.

On déclare une fonction \texttt{deplacement(Segment1, Segment2, Nœud)} (déplacement de \texttt{Segment2} sur \texttt{Nœud} par rapport à \texttt{Segment1}) que l'on définit comme suit :
~ \\
Pour tout segment $p \in \pre(s)$:
\begin{itemize}
\item {Si $p$ et $s$ sont sur le même nœud $N \neq N_d$ :

On considère $E_{p}$ l'ensemble des places d'entrées de $p$.
On note $t_p$ la durée de la transition de $p$.

\begin{itemize}
\item Si $p$ n'est pas précédé par un $\alpha$-segment, on en crée un de la manière décrite précédemment, en \ref{section:alphasegment}.
\item On rajoute en sortie de cet $\alpha$-segment une place $e_s$.
\end{itemize}
\vspace{1em}

On crée ensuite une transition $T_s$ de durée $t_p - \delta_{N \rightarrow N_d}$ (avec un minimum de 0), à laquelle on rajoute en entrée la place $e_s$ créée précédemment. On déplace $s$ de telle sorte que la place de $\shared(p, s)$ soit maintenant en sortie de $T_s$.
L'ensemble composé des places $e_s$, de $T_s$, et de la place d'entrée de $s$ est un $\beta$-segment : c'est le segment qui effectue la bufferisation.

Néanmoins, la place qui était en sortie de la transition de $p$ et partagée avec $s$ est conservée, comme dans la méthode précédente. De cette manière, on peut garder la position originale pour l'opération de recombinaison.
~ \\

C'est le résultat de \texttt{deplacement(p, s, Nd)} lorsque $s$ et $p$ ont même nœud initial.

Un exemple partiel est fourni en figure \ref{fig:deplacementForm2}.
}
\\
\item{ Si $p$ est sur un nœud $N_1$, $s$ sur un nœud $N_2 \neq N_1$, et que $N_d \neq N_1$, comme il y a déjà eu une opération de déplacement, il suffit d'ajuster le temps de la transition $T_s$, qui vaut $t_p - \delta_{N_1 \rightarrow N_2}$ : on ajoute $\delta_{N_1 \rightarrow N_2}$ et on enlève le nouveau délai entre les deux nœuds, $\delta_{N_1 \rightarrow N_d}$.

~ \\
C'est le résultat de \texttt{deplacement(p, s, Nd)} lorsque $s$ et $p$ n'ont pas même nœud initial.
}
\\
\item Si $p$ est sur un nœud $N$, et que $N_d = N$, on effectue une opération de recombinaison : toutes les places de sortie de $p$ qui avaient été préservées dans le premier cas de déplacement sont fusionnées avec les places réelles, et le $\beta$-segment correspondant à ce déplacement est supprimé. S'il n'est plus suivi d'aucun $\beta$-segment, on peut aussi supprimer l'$\alpha$-segment précédant $p$.
~ \\
~ \\
C'est le résultat de \texttt{deplacement(p, s, Nd)} lorsque $N_d$ est le nœud de $p$.
\end{itemize}

On applique ensuite la même transformation entre $s$ et les segments appartenant à $\post(s)$.

\subparagraph{Choix de répartition}
Deux choix sont possibles pour la répartition : 

\begin{itemize}
\item Exécuter $T_s$, la transition du $\beta$-segment, sur le nœud d'origine.
\item Exécuter $T_s$ sur le nœud ou est déplacé $s$.
\end{itemize}

Dans le premier cas, on attend jusqu'au dernier moment pour envoyer le message, dans le second, on l'envoie au plus tôt. Selon les applications, les deux approches peuvent avoir leur intérêt.

Par exemple, on peut préférer la deuxième approche si le lien de connexion est peu stable. On peut préférer la première approche si des évènements venaient à modifier le réseau de Petri pendant l'exécution.


\begin{figure}[H]
\centering
\begin{tikzpicture}[node distance=1.3cm,>=stealth,bend angle=45,auto]
  \tikzstyle{segment}=[]
  \tikzstyle{place}=[circle,thick,draw=blue!75,fill=blue!20,minimum size=6mm]
  \tikzstyle{red place}=[place,draw=red!75,fill=red!20]
  \tikzstyle{transition}=[rectangle,thick,draw=black!75,
  			  fill=black!20,minimum size=4mm]


  \begin{scope}
  \node[segment](I){In};
  \node [place] (A) [right of=I] {$A_1$};
  \node [place] (A2) [below of=A] {$A_2$};
  \node[segment](I2) [left of=A2]{In 2};
  \node [transition] (Ta) [right of=A, label=$t_A$] {$T_A$};
  \node [place] (B) [right of=Ta] {$B_1$};
  \node [place] (A3) [above of=B] {$A_3$};
  \node [transition] (Tb) [right of=B, label=$t_B$] {$T_B$};
  \node [place] (C) [right of=Tb] {$C_1$};
  \node[segment](O) [right of=C]{Sortie};
  
  \path
  		(I) edge[post] node {} (A)
  		(I2) edge[post] node {} (A2)
  		(A) edge[post] node {} (Ta)
  		(A2) edge[post] node{} (Ta)
  		(Ta) edge[post] node {} (B)
  			 edge[post] node {} (A3)
  		(B) edge[post] node {} (Tb)
  		(Tb) edge[post] node {} (C)
  		(C) edge[post] node {} (O);
  \end{scope}
\end{tikzpicture}
  
\vspace{1em}
Devient
\vspace{1em}
  
\begin{tikzpicture}[node distance=1.3cm,>=stealth,bend angle=45,auto]
  \tikzstyle{segment}=[]
  \tikzstyle{place}=[circle,thick,draw=blue!75,fill=blue!20,minimum size=6mm]
  \tikzstyle{red place}=[place,draw=red!75,fill=red!20]
  \tikzstyle{transition}=[rectangle,thick,draw=black!75,
  			  fill=black!20,minimum size=4mm]
  \tikzstyle{red}=[draw=red!75,fill=red!20]
 \tikzstyle{green}=[draw=green!75,fill=green!20]

  \begin{scope}
  %alpha-sgmt
  \node[segment](I){In};
  \node [place] (Aalpha) [red, right of=I] {$A_1$};
  \node [transition] (Talpha) [red,right of=Aalpha, label=$0$] {$T_{\alpha A}$};

  \node[segment](I2) [left of=A2]{In 2};
  \node [place] (Aalpha2) [red, below of=Aalpha] {$A_2$};
%  \node [transition] (Talpha2) [red,below of=Talpha, label=$0$] {};

  \node [place] (A) [right of=Talpha] {$A_i$};
  
  \node [transition] (Ta) [right of=A, label=$t_A$] {$T_A$};
  \node [place,dotted] (Bprime) [right of=Ta] {$B'_1$};
  \node [place] (A3) [above of=Bprime] {$A_3$};
  \node [place] (AS) [green,below of=A] {$A_B$};
  \node [transition] (TS) [green,right of=AS, label=$t_A - x$] {$T_{\beta B}$};
  
  \node [place] (B) [right of=TS] {$B_1$};
  \node [transition] (Tb) [right of=B, label=$t_B$] {$T_B$};
  \node [place] (C) [right of=Tb] {$C_1$};
  \node[segment](O) [right of=C]{Sortie};
  
  \path
  		(I) edge[post] node {} (Aalpha) 
  		(I2) edge[post] node {} (Aalpha2)
  		(Aalpha) edge[post,red] node {} (Talpha)
  		(Talpha) edge[red,post] node {} (A)
  		         edge[red,post] node {} (AS)
  		(Aalpha2) edge[post,red] node {} (Talpha)
%  		(Talpha2) edge[red,post] node {} (A)
%  		         edge[red,post] node {} (AS)
  		(A) edge[post] node {} (Ta) 
  		(Ta) edge[post] node {} (Bprime) 
  			 edge[post] node {} (A3)
  		(AS) edge[green,post] node {} (TS) 
  		(TS) edge[green,post] node {} (B)         
  		(B) edge[post] node {} (Tb)
  		(Tb) edge[post] node {} (C)
  		(C) edge[post] node {} (O);
  \end{scope}
\end{tikzpicture}
\caption{Exemple : seconde méthode de déplacement}
\label{fig:deplacementForm2}
\end{figure}

Sur la figure \ref{fig:deplacementForm2}, on n'a appliqué \texttt{deplacement} qu'au segment précédent, et non au segment suivant.

On y montre comment on déporte l'exécution du segment $B(B_1, T_B, C_1)$.

\paragraph{Duplication}
L'opération de duplication consiste à faire s'exécuter un nd-segment $s$ sur plusieurs nœuds de $G$ (distincts?). Il faut donc le recopier.

Cette opération revient à la suite d'opérations suivantes : 
\begin{enumerate}
\item Dupliquer localement $s$ : \\
		On crée un nouveau segment similaire à $s$ (même nombre de places, transitions, même durée, mêmes processus et évènements).
		
		Des arcs sont créés dans les segments précédant et suivant $s$ pour répliquer le comportement de $s$, en un nouveau segment $s'$.
\item Déplacer $s'$ avec une des méthodes vues précédemment.
\end{enumerate}

Un cas de duplication pour la seconde méthode est donné en exemple en figure \ref{fig:duplicationEtRecoll}.

\section{Méthode par coloration}
\chaptermark{Coloration}
Cette formalisation utilise la notion de coloration du réseau.

Chaque nœud de $G$ se voit associer une couleur distincte.

\subsection{Généralités}
Les places et les transitions du réseau de Petri peuvent aussi être colorés avec ces mêmes couleurs. Elles peuvent avoir plusieurs couleurs.
À l'état initial, le réseau entier est coloré avec la couleur du premier nœud.

Si une transition est sensibilisée (elle possède un jeton de couleur $c$ dans toutes ses places d'entrée), elle adopte le comportement suivant : 

Pour chaque place en sortie :
\begin{enumerate}
\item Elle compare la couleur $c$ avec la couleur $d$ de la place. 
\item Soit $N_1$ le nœud qui possède la couleur $c$, $N_2$ le nœud qui possède la couleur $d$. Soit $\delta$ le délai entre $N_1$ et $N_2$. Soit $t$ la durée de la transition. La transition s'exécute pour ce jeton avec la durée $t - \delta$, avec un minimum de 0, en gardant en mémoire les jetons qui doivent s'exécuter plus tard.
\end{enumerate}

\subsubsection{Exécution d'une transition}
L'exécution d'une transition présente de plus la particularité suivante : comme une place peut avoir plusieurs couleurs, si une transition doit placer un jeton dans une place, elle y dépose pour chaque couleur de la place, un jeton de cette couleur. 
Une transition est exécutée par le nœud de sa couleur.

\subsection{Opérations sur segment}
\subsubsection{Déplacement}
On change la couleur de la transition et des places d'entrée du segment que l'on désire déplacer.

\subsubsection{Duplication}
Cette opération est similaire au déplacement : au lieu de changer la couleur, on en ajoute une. Si on désire enlever un nœud, il suffit de supprimer une couleur.

Un exemple est donné en figure \ref{fig:repartCouleur}.

\subsection{Problèmes}
Cette méthode présente plusieurs faiblesses : 
\begin{itemize}
	\item Elle change l'algorithme d'exécution, ce qui peut poser des problèmes de compatibilité.
	\item L'utilisation de la couleur fait que cette dernière ne peut pas être utilisée facilement dans d'autres extensions du formalisme \ac{OSSIA}, comme par exemple dans le cas du traitement de données\cite{arias2014modelling}.
	\item Il n'y a pas de moyen facile de re-synchroniser plusieurs machines dans le cas de la duplication.
\end{itemize}

\begin{figure}[H]
	\centering
	\begin{tikzpicture}[node distance=1.3cm,>=stealth,bend angle=45,auto]
  \tikzstyle{segment}=[]
  \tikzstyle{place}=[circle,thick,minimum size=6mm]
  \tikzstyle{transition}=[rectangle,thick,minimum size=4mm]

  \tikzstyle{blue}=[draw=blue!75,fill=blue!20]
  \tikzstyle{red}=[draw=red!75,fill=red!20]
  \tikzstyle{black}=[draw=black!75,fill=black!20]

  \tikzstyle{tblue}=[draw=blue!100,fill=blue!75]
  \tikzstyle{tred}=[draw=red!100,fill=red!75]
  \tikzstyle{tblack}=[draw=black!100,fill=black!75]

  \begin{scope}
  \node[segment](I){In};
  \node [blue,place,colored tokens={blue}] (A) [right of=I,label=$A_1$] {};
  \node [blue,place] (A2) [below of=A,label=$A_2$,colored tokens={blue}] {};
  \node[segment](I2) [left of=A2]{In 2};
  \node [blue,transition] (Ta) [right of=A, label=$T_A$] {};
  \node [blue,place] (B) [right of=Ta,label=$B_1$] {};
  \node [blue,place] (A3) [above of=B,label=$A_3$] {};
  \node [blue,transition] (Tb) [right of=B, label=$T_B$] {};
  \node [blue,place] (C) [right of=Tb,label=$C_1$] {};
  \node[segment](O) [right of=C]{Sortie};
  
  \path
  		(I) edge[post] node {} (A)
  		(I2) edge[post] node {} (A2)
  		(A) edge[post] node {} (Ta)
  		(A2) edge[post] node{} (Ta)
  		(Ta) edge[post] node {} (B)
  			 edge[post] node {} (A3)
  		(B) edge[post] node {} (Tb)
  		(Tb) edge[post] node {} (C)
  		(C) edge[post] node {} (O);
  \end{scope}
\end{tikzpicture}

\vspace{1em}
On déplace le segment $B(B_1, T_B, C_1) $ sur le nœud de couleur noire: 
\vspace{1em}

\begin{tikzpicture}[node distance=1.3cm,>=stealth,bend angle=45,auto]
  \tikzstyle{segment}=[]
  \tikzstyle{place}=[circle,thick,minimum size=6mm]
  \tikzstyle{transition}=[rectangle,thick,minimum size=4mm]

  \tikzstyle{blue}=[draw=blue!75,fill=blue!20]
  \tikzstyle{red}=[draw=red!75,fill=red!20]
  \tikzstyle{black}=[draw=black!75,fill=black!20]

  \tikzstyle{tblue}=[draw=blue!100,fill=blue!75]
  \tikzstyle{tred}=[draw=red!100,fill=red!75]
  \tikzstyle{tblack}=[draw=black!100,fill=black!75]

  \begin{scope}
  \node[segment](I){In};
  \node [blue,place,colored tokens={blue}] (A) [right of=I,label=$A_1$] {};
  \node [blue,place] (A2) [below of=A,label=$A_2$,colored tokens={blue}] {};
  \node[segment](I2) [left of=A2]{In 2};
  \node [blue,transition] (Ta) [right of=A, label=$T_A$] {};
  \node [black,place] (B) [right of=Ta,label=$B_1$] {};
  \node [blue,place] (A3) [above of=B,label=$A_3$] {};
  \node [black,transition] (Tb) [right of=B, label=$T_B$] {};
  \node [blue,place] (C) [right of=Tb,label=$C_1$] {};
  \node[segment](O) [right of=C]{Sortie};
  
  \path
  		(I) edge[post] node {} (A)
  		(I2) edge[post] node {} (A2)
  		(A) edge[post] node {} (Ta)
  		(A2) edge[post] node{} (Ta)
  		(Ta) edge[post] node {} (B)
  			 edge[post] node {} (A3)
  		(B) edge[post] node {} (Tb)
  		(Tb) edge[post] node {} (C)
  		(C) edge[post] node {} (O);
  \end{scope}
\end{tikzpicture}

\vspace{1cm}
Puis, après que $T_A - \delta_{Bleu \rightarrow Noir}$ soit écoulé : 
\vspace{1cm}


\begin{tikzpicture}[node distance=1.3cm,>=stealth,bend angle=45,auto]
  \tikzstyle{segment}=[]
  \tikzstyle{place}=[circle,thick,minimum size=6mm]
  \tikzstyle{transition}=[rectangle,thick,minimum size=4mm]

  \tikzstyle{blue}=[draw=blue!75,fill=blue!20]
  \tikzstyle{red}=[draw=red!75,fill=red!20]
  \tikzstyle{black}=[draw=black!75,fill=black!20]

  \tikzstyle{tblue}=[draw=blue!100,fill=blue!75]
  \tikzstyle{tred}=[draw=red!100,fill=red!75]
  \tikzstyle{tblack}=[draw=black!100,fill=black!75]

  \begin{scope}
  \node[segment](I){In};
  \node [blue,place] (A) [right of=I,label=$A_1$] {};
  \node [blue,place] (A2) [below of=A,label=$A_2$] {};
  \node[segment](I2) [left of=A2]{In 2};
  \node [blue,transition] (Ta) [right of=A, label=$T_A$,colored tokens={blue}] {};
  \node [black,place] (B) [right of=Ta,label=$B_1$,colored tokens={black}] {};
  \node [blue,place] (A3) [above of=B,label=$A_3$] {};
  \node [black,transition] (Tb) [right of=B, label=$T_B$] {};
  \node [blue,place] (C) [right of=Tb,label=$C_1$] {};
  \node[segment](O) [right of=C]{Sortie};
  
  \path
  		(I) edge[post] node {} (A)
  		(I2) edge[post] node {} (A2)
  		(A) edge[post] node {} (Ta)
  		(A2) edge[post] node{} (Ta)
  		(Ta) edge[post] node {} (B)
  			 edge[post] node {} (A3)
  		(B) edge[post] node {} (Tb)
  		(Tb) edge[post] node {} (C)
  		(C) edge[post] node {} (O);
  \end{scope}
\end{tikzpicture}

\vspace{1cm}
Enfin, après que $T_A$ soit écoulé (si $T_B-\delta_{Noir \rightarrow Bleu}$ ne l'est pas encore) : 
\vspace{1cm}

\begin{tikzpicture}[node distance=1.3cm,>=stealth,bend angle=45,auto]
  \tikzstyle{segment}=[]
  \tikzstyle{place}=[circle,thick,minimum size=6mm]
  \tikzstyle{transition}=[rectangle,thick,minimum size=4mm]

  \tikzstyle{blue}=[draw=blue!75,fill=blue!20]
  \tikzstyle{red}=[draw=red!75,fill=red!20]
  \tikzstyle{black}=[draw=black!75,fill=black!20]

  \tikzstyle{tblue}=[draw=blue!100,fill=blue!75]
  \tikzstyle{tred}=[draw=red!100,fill=red!75]
  \tikzstyle{tblack}=[draw=black!100,fill=black!75]

  \begin{scope}
  \node[segment](I){In};
  \node [blue,place] (A) [right of=I,label=$A_1$] {};
  \node [blue,place] (A2) [below of=A,label=$A_2$] {};
  \node[segment](I2) [left of=A2]{In 2};
  \node [blue,transition] (Ta) [right of=A, label=$T_A$] {};
  \node [black,place] (B) [right of=Ta,label=$B_1$,colored tokens={black}] {};
  \node [blue,place] (A3) [above of=B,label=$A_3$,colored tokens={blue}] {};
  \node [black,transition] (Tb) [right of=B, label=$T_B$] {};
  \node [blue,place] (C) [right of=Tb,label=$C_1$] {};
  \node[segment](O) [right of=C]{Sortie};
  
  \path
  		(I) edge[post] node {} (A)
  		(I2) edge[post] node {} (A2)
  		(A) edge[post] node {} (Ta)
  		(A2) edge[post] node{} (Ta)
  		(Ta) edge[post] node {} (B)
  			 edge[post] node {} (A3)
  		(B) edge[post] node {} (Tb)
  		(Tb) edge[post] node {} (C)
  		(C) edge[post] node {} (O);
  \end{scope}
\end{tikzpicture}
	\caption{Exemple : répartition par couleur} 
	\label{fig:repartCouleur}
\end{figure}

\newpage
\section{Comparaison entre les méthodes}
Dans cette partie, nous faisons une brève comparaison entre les méthodes vues jusqu'à présent, en considérant les qualités et les défauts de chacune, en \cref{tblcompMethods}.

\begin{table}[h]
	\centering
	\tabulinesep=3pt
	\begin{tabu}  to \linewidth {X[0.1,c]XXX}
		& Méthode 1 & Méthode 2 & Méthode 3 \\
		\toprule
		\multirow{5}{*}{{\textbf{\huge $+$}}}
			& Transformation simple & Conserve les nd-segments & Ne change pas la topologie \\
			& & Passe bien à l'échelle & Passe bien à l'échelle \\
			& & Permet effets particuliers (cf. \ref{section:canon}) & Théorie simple \\
			& & Permet rétropropagation (cf. \ref{section:retropropag}) & \\
			& & Permet synchronisation (cf. \ref{fig:duplicationEtRecoll}) &  \\
			\hline
		\multirow{3}{*}{{\textbf{\huge $-$}}} 
			& Passe mal à l'échelle & Transformation complexe & Change l'algorithme \\
			& nd-segments découpés  & Introduit de nouveaux segments & Bloque les couleurs  \\
			& Boucles ingérables & & 
	\end{tabu}
	\caption{Points positifs et négatifs des méthodes}
	\label{tblcompMethods}
\end{table}

En raison du nombre de points positifs de la méthode 2, et de la durée du stage, c'est celle qui a été principalement retenue pour vérifications ultérieures.

\section{Améliorations possibles sur seconde formalisation}
\subsection{Politiques d'\texorpdfstring{$\alpha$-transitions}~}
\label{section:canon}
La sémantique des $\alpha$-transitions peut être modifiée afin de fournir plusieurs politiques.

Par exemple : 
\begin{itemize}
\item Envoyer un jeton vers toutes ses branches en sortie.
\item N'envoyer un jeton que vers certaines branches choisies.
\item Envoyer des jetons vers une ou des branches choisies au hasard.
\item Envoyer les jetons en différé, pour créer un effet de canon par exemple.
\item Garder le fait qu'un jeton soit passé en mémoire et le renvoyer plus tard (pour une reprise sur erreur, par exemple).
\end{itemize}

Cela offre plus de latitude au compositeur. Néanmoins le réseau de Petri n'est plus "pur", on modifie l'algorithme d'exécution d'une transition.


\subsection{Rétropropagation du buffering}
\label{section:retropropag}
Un processus a été conçu pour essayer de régler le problème des latences trop grandes entre nœuds.

Il est utile dans le cas où l'on n'a pas de point d'interaction entre différents segments successifs.

\begin{mydef}[Point d'interaction]
	Un point d'interaction est un endroit où l'exécution du réseau de Petri est soumise à influence extérieure : condition, déclenchement interactif, etc. Une définition complète est donnée dans \cite[p. 148]{allombert2009aspects}.
\end{mydef}

\paragraph{Processus}
Soient deux segments $p$ et $s$ tels que $s$ soit successif à $p$. On désire déplacer $s$, mais la latence entre les deux nœuds est supérieure à la durée de la transition de $p$ : la différence $t_p - \delta$ est négative.


Pour tout segment $u \in \pre(p)$ : 
\begin{enumerate}
\item S'il n'existe pas, créer un $\alpha$-segment le précédant.
\item Rajouter une place en sortie de chaque $\alpha$-segment créé à l'étape précédente.
\item Supprimer l'arc précédant le $\beta$-segment correspondant à $s$ créé lors de \texttt{deplacement(p, s, \dots)}.
\item Créer un arc entre toutes les places créées et ce $\beta$-segment.
\end{enumerate}
Soit $t_p - \delta$ la durée du $\beta$-segment. En raison de la latence élevée, $t_p - \delta$ est négatif. Soient $t_{u_1}, ..., t_{u_n}$ les durées des segments de $\pre(p)$.

On remplace $t_p - \delta$ par $\max(t_{u_1}, ..., t_{u_n}) + t_p - \delta$.

Si la durée obtenue est toujours négative, on peut réitérer ce processus jusqu'à une de ces deux possibilités : 

\begin{itemize}
\item Obtention d'une durée positive ou nulle.
\item Arrivée à un point d'interactivité.
\end{itemize}

\subsection{Utilisation de la duplication pour un backup}
Le backup consiste à être en possession de plusieurs nœuds exécutant la même partie du réseau, mais un seul exécute réellement les processus contenus entre les places.

Cela permet, si un des nœuds rencontre un problème informatique, d'avoir un autre nœud qui prenne le relai.

Deux approches sont possibles :
\begin{itemize}
\item Une, déjà évoquée, est d'avoir l'opération qui repart depuis le début du déplacement ($\alpha$-segment).
\item L'autre serait d'essayer de repartir au plus proche, c'est-à-dire :  
\begin{itemize}
\item Soit replacer un jeton au début de la dernière transition s'étant exécutée. Cela implique un léger retour en arrière.
\item Soit attendre que la dernière transition qui était en cours d'exécution se termine. Cela implique une impression de "mise en pause".
\end{itemize}
\end{itemize}

Dans les deux cas présentés ci-dessus, il sera impossible d'échapper à un effet de saut lors de la reprise pour certains processus. Mais une granularité plus fine (c'est-à-dire au niveau des processus et du temps réel) n'est pas possible avec le formalisme des réseaux de Petri.

Cependant, il est simple de l'implémenter : il suffit d'effectuer une duplication au niveau du réseau de Petri, et de donner aux nœuds du graphe une propriété avec comme états posibles \textit{activé} ou \textit{désactivé}. Si un nœud ne répond plus et est dupliqué à des fins de backup, un autre nœud \textit{active} la branche censée prendre le relai; ce nœud émet alors instantanément des signaux à ses processus.

De plus, l'idéal serait de pouvoir mettre en sourdine (\textit{to mute}) uniquement les sorties matérielles de l'ordinateur (ex. : sortie audio, vidéo-projecteur). En effet, un processus externe peut utiliser un effet de feedback; une transition sans interruptions entre une machine ayant planté et la machine de backup ne fonctionnera correctement que si toutes les étapes précédentes ont été calculées par la machine de backup.

Enfin, pour des questions d'ergonomie, il serait intéressant d'avoir un concept de groupes de nœuds, censés posséder les mêmes tâches, et d'associer à chaque segment un ou plusieurs groupes. La détection d'un nœud "mort" se ferait dans le groupe par envoi de messages réguliers, et une des machines au hasard du groupe reprendrait les tâches laissées en plan.

\subsection{Cas du changement de vitesse}
Le formalisme~\cite{allombert2010virage} permet de changer la vitesse d'exécution, en multipliant les durées des transitions par un facteur numérique.

Pour conserver la correction du formalisme décrit ici, il ne faut pas multiplier le facteur de décalage entre les nœuds, car il ne dépend pas de la vitesse d'exécution du morceau: il s'agit d'une caractéristique physique du réseau.

Il serait par exemple possible d'affecter une fonction affine de la forme $av + b$ aux transitions, ou $a$ est la durée de la transition canonique, $v$ le coefficient multiplicateur de vitesse compris entre $0.5$ et $2$, et $b$ le délai. 

\subsection{Simplifications}
\subsubsection{$\alpha$-transitions}
\begin{itemize}
\item Si un $\alpha$-segment est en sortie d'un autre, ils peuvent être fusionnés : les places de sortie du segment suivant sont reliés à la transition du précédent. Cela ne change rien : dans les deux cas les délais des transitions sont nuls.

\item De même, si un $\alpha$-segment suit directement un $\beta$-segment, on peut supprimer l'$\alpha$-segment et transférer les places qui étaient à sa sortie, en sortie du $\beta$-segment.
\end{itemize}
\subsubsection{Connaissance de la topologie du réseau de Petri}
On peut essayer de faire en sorte que les nœuds n'aient connaissance que des segments qu'ils possèdent et sont censés exécuter, et ce afin d'éviter la duplication.

Ils doivent au moins posséder : 
\begin{itemize}
\item Leurs places d'entrée, pour savoir quand les transitions sont sensibilisées.
\item Leurs transitions.
\item Leurs places de sortie qui correspondent à une fin de processus.
\end{itemize}

Les autres envois de jetons peuvent être faits par le réseau.

\paragraph{Prise de connaissance dynamique ?}
Tant que l'on se situe avant l'exécution d'un segment par une machine, une autre machine peut lui transférer ce segment par le réseau.

Il faut néanmoins s'assurer qu'elle reçoive le / les segments avant leurs débuts, en calculant la latence entre les nœuds, et informer les autres nœuds du nouveau responsable de ce segment, pour qu'ils sachent ou envoyer leurs jetons.

\subsubsection{Connaissance de l'état du réseau de Petri}
Si l'on opte pour une connaissance globale, est-il néanmoins nécessaire que chaque nœud ait connaissance de la position de chaque jeton à chaque instant et fasse évoluer ses propres jetons de son côté ? 

Cela pose problème dans :
\begin{itemize}
\item Le cas des points d'interaction, ce qui engendre parfois une surcharge au niveau du réseau, si par exemple la valeur d'une variable sur laquelle porte une condition doit être transmise à tous les nœuds.
\item Le cas où une machine aurait un ralentissement : il y aurait une incohérence entre la position de son jeton et la position du même jeton dans les autres machines.
\end{itemize}

Une solution serait d'avoir une machine maîtresse, dont l'état est prioritaire sur tous les autres. 

Ou alors de ne pas avoir de réplication d'état : cela veut dire que la plupart des machines seront toujours dans un état incohérent vis-à-vis de la sémantique des réseaux de Petri (car des jetons vont apparaître et disparaître au gré des envois par le réseau), mais le fonctionnement est plus simple à gérer et provoquerait moins de problèmes sur le long terme.

\subsection{Déplacement de groupes de segments}
Si on désire déplacer plusieurs segments adjacents en même temps, il suffit d'appliquer l'opération de déplacement aux segments possédant des places d'entrée ou de sortie communes à un segment ne faisant pas partie de ce groupe. En effet, seuls ceux-là sont sensibles au délai entre les nœuds.

\subsection{Approche optimisée pour implémentation}
Comme i-score précompile les réseaux de Petri à chaque lecture, et ne les génère pas de façon dynamique, il serait possible d'appliquer le processus "inverse" de celui décrit, pour avoir un nombre de calculs minimaux à réaliser : 

\begin{enumerate}
\item Affecter à chaque segment le nœud sur lequel il est censé se trouver.
\item Pour chaque paire de segments adjacents et n'étant pas sur le même nœud, effectuer une opération de déplacement.
\end{enumerate}

Sans cette optimisation, il y aurait par exemple des cas de déplacements suivis de recombinaisons inutiles.
\section{Preuves et vérifications}
Cette section s'adresse encore spécifiquement au cas de la seconde formalisation présentée tantôt.

\subsection{Correction de la solution par rapport au problème} 
Il convient d'abord de définir ce que signifie une exécution correcte.
On peut parcourir le réseau (en utilisant par exemple un parcours en largeur), et affecter à chaque transition un instant ou elle doit s'exécuter, jusqu'à ce que toutes les transitions non-interactives aient un temps affecté. Puis, à chaque activation de transition interactive, on peut affecter un temps aux transitions qui la suivent (en faisant un sous-parcours), et ce jusqu'à la fin de l'exécution du morceau.

On considère une horloge globale, initialisée à 0, et que l'on démarre en même temps que l'exécution du scénario.

Une exécution correcte est une exécution ou toutes les transitions commencent à s'exécuter effectivement à l'instant qui leur est affecté (le cas réel a une précision d'une milliseconde).

\begin{myth}
Quand on déplace un segment, le réseau s'exécute correctement (dans la contrainte d'un délai fixe).
\end{myth}

Soit $p$ un segment, et soit $s$ un segment attenant à $p$ que l'on déplace sur un nœud différent de celui de $s$ par le processus décrit précédemment.

Comme les nœuds de $p$ et $s$ sont séparés par un délai de transmission d'information $\delta$, quand la $\beta$-transition s'exécute, l'information arrive en place d'entrée de $D$ à $t$ car $t - \delta + \delta = t$.   

\subsection{Inversibilité du processus}
\begin{myth}
	L'opération de recombinaison est inverse de celle du déplacement.
\end{myth}
Soit $p$ un segment, et soit $s$ un segment attenant à $p$ que l'on déplace sur un nœud différent de celui de $p$ par le processus décrit précédemment.

$p$ est donc précédé d'un $\alpha$-segment et est suivi d'un $\beta$-segment menant à $s$
Par construction, re-déplacer $s$ sur le nœud ou se situe $p$ revient à supprimer le $\beta$-segment, et replacer les entrées de $s$ qui sont aussi des sorties de $p$ en sortie de la transition de base de $p$. S'il n'y a plus de $\beta$-segments suivant l'$\alpha$-segment, on peut le supprimer.

On retrouve alors la configuration initiale.

\subsection{Vérification dans le cas des boucles}
On désire déplacer un des deux segments de la boucle sur un autre nœud.

Les boucles sont un cas particulier : elles possèdent une condition en sortie.
Une condition, selon le formalisme mis en place par Antoine Allombert (TODO citer page), se manifeste par une place qui se trouve en entrée de plusieurs transitions, ce que nous avions interdit jusqu'alors. La condition se manifeste donc en pratique par la transition qui est tirée si cette condition est vraie.

Deux possibilités s'offrent alors : 
\begin{itemize}
\item Permettre aux places d'entrée d'avoir plusieurs sorties, ce qui leur donne plus de sens. C'est ce qui est choisi ici. On n'est plus alors dans le cas d'un graphe d'évènements.
\item Considérer qu'une place est équivalente à un segment de durée nulle, possédant autant de places d'entrée que d'arcs d'entrée de la place, et autant de places de sortie que d'arcs de sortie de la place. Cela déporte la logique des conditions dans la transition de ce nouveau segment. Mais sans conditions, ce n'est pas strictement équivalent, car on aurait une duplication de jetons supplémentaire.
\end{itemize}

Nous avons dans la figure \ref{fig:verifBoucles} un exemple de ce qui se passe si l'on désire déplacer une transition d'une boucle sur une autre machine. (Les deux nœuds sont notés $N$ et $N_d$, en suivant la notation employée jusque là).

\begin{figure}[H]
\centering
\begin{tikzpicture}[node distance=1.3cm,>=stealth,bend angle=45,auto]
  \tikzstyle{segment}=[]
  \tikzstyle{place}=[circle,thick,minimum size=6mm]
  \tikzstyle{transition}=[rectangle,thick,minimum size=4mm]

  \tikzstyle{blue}=[draw=blue!75,fill=blue!20]
  \tikzstyle{red}=[draw=red!75,fill=red!20]
  \tikzstyle{black}=[draw=black!75,fill=black!20]

  \tikzstyle{tblue}=[draw=blue!100,fill=blue!75]
  \tikzstyle{tred}=[draw=red!100,fill=red!75]
  \tikzstyle{tblack}=[draw=black!100,fill=black!75]

  \begin{scope}
  \node[segment](I){In};
  \node [blue,place] (A) [right of=I,label=$A$] {};
  \node [blue,transition] (Ta) [right of=A, label=$T_A$] {};
  \node [blue,place] (B) [right of=Ta,label=$B$] {};
  \node [blue,transition] (Tb) [above of=Ta, label=$T_B$] {};
  \node[segment](O) [right of=B]{Sortie};
  
  \path
  		(I) edge[post] node {} (A)
  		(A) edge[post] node {} (Ta)
  		(Ta) edge[post] node {} (B)
  		(B) edge[post] node {} (Tb)
  		(Tb) edge[post] node {} (A)
  		(B) edge[post] node {} (O);
  \end{scope}
\end{tikzpicture}

\vspace{1em}
Devient
\vspace{1em}

\begin{tikzpicture}[node distance=1.3cm,>=stealth,bend angle=45,auto]
  \tikzstyle{segment}=[]
  \tikzstyle{place}=[circle,thick,minimum size=6mm]
  \tikzstyle{transition}=[rectangle,thick,minimum size=4mm]

  \tikzstyle{blue}=[draw=blue!75,fill=blue!20]
  \tikzstyle{red}=[draw=red!75,fill=red!20]
  \tikzstyle{black}=[draw=black!75,fill=black!20]
  \tikzstyle{green}=[draw=green!75,fill=green!20]

  \tikzstyle{tblue}=[draw=blue!100,fill=blue!75]
  \tikzstyle{tred}=[draw=red!100,fill=red!75]
  \tikzstyle{tblack}=[draw=black!100,fill=black!75]

  \begin{scope}
  \node[segment](I){In};
  \node[red,place] (Aalpha) [right of=I,label=$A_\alpha$] {};
  \node[red,transition] (Talpha) [right of=Aalpha, label=$T_\alpha$] {}; 
  \node[green,place] (Aalpha0) [right of=Talpha,label=$A_{\alpha0}$] {};
  
  \node[blue,place] (A) [above of=Aalpha0,label=$A$] {};
  \node[blue,transition] (Ta) [right of=A, label=$T_A$] {};
  \node[dotted,blue,place] (B0) [right of=Ta,label=$B_0$] {};
 
 
  \node[green,transition] (Tbeta) [right of=Aalpha0, label=$T_\beta$] {};
  \node[blue,place] (B) [right of=Tbeta,label=$B$] {};
  \node[blue,transition] (Tb) [below of=Tbeta, label=$T_B$] {};
  \node[segment](O) [right of=B]{Sortie};
  
  \path
  		(I) edge[post] node {} (Aalpha)
  		(Aalpha) edge[post] node {} (Talpha)
  		(Talpha) edge[post] node {} (Aalpha0)
  				 edge[post] node {} (A)
  		(Aalpha0) edge[post] node {} (Tbeta)
  		(Tbeta) edge[post] node {} (B)
  		(A) edge[post] node {} (Ta)
  		(Ta) edge[post] node {} (B0)
  		(B) edge[post] node {} (Tb)
  		(Tb) edge[post] node {} (Aalpha)
  		(B) edge[post] node {} (O);
  \end{scope}
\end{tikzpicture}

\vspace{1em}
Enfin :
\vspace{1em}


\begin{tikzpicture}[node distance=1.3cm,>=stealth,bend angle=45,auto]
  \tikzstyle{segment}=[]
  \tikzstyle{place}=[circle,thick,minimum size=6mm]
  \tikzstyle{transition}=[rectangle,thick,minimum size=4mm]

  \tikzstyle{blue}=[draw=blue!75,fill=blue!20]
  \tikzstyle{red}=[draw=red!75,fill=red!20]
  \tikzstyle{black}=[draw=black!75,fill=black!20]
  \tikzstyle{green}=[draw=green!75,fill=green!20]

  \tikzstyle{tblue}=[draw=blue!100,fill=blue!75]
  \tikzstyle{tred}=[draw=red!100,fill=red!75]
  \tikzstyle{tblack}=[draw=black!100,fill=black!75]

  \begin{scope}
  % Début
    \node[segment](I){In};
    \node[red,place] (Aalpha) [right of=I] {$A_\alpha$};
    \node[red,transition] (Talpha) [right of=Aalpha, label=$0$] {$T_\alpha$}; 
  
  % Branche centrale
    \node[green, place] (Abeta2) [right of=Talpha] {$A_{\beta}$};
    \node[green,transition] (Tbeta) [right of=Abeta2, label=$t_A-t_1$] {$T_\beta$};
    
    \node[red,place] (Balpha) [right of=Tbeta] {$B_\alpha$};
    \node[red, transition] (Talpha2) [right of=Balpha, label=$0$] {$T_\alpha$};
    
    \node[blue,place] (B) [right of=Talpha2] {$B$};
    \node[blue, transition] (TB) [right of=B, label=$t_B$] {$T_B$};
    \node[blue, dotted, place] (AFake) [right of=TB] {$A_0$};
    
  % Branche du bas
    \node[green, place] (Bbeta) [below of=B] {$B_\beta$};
    \node[green,transition] (Tbeta2) [right of=Bbeta,label=$t_B-t_2$] {$T_{\beta2}$};
    \node[segment](O) [above of=TB]{Sortie};
  
  % Branche supérieure
    \node[blue,place] (A) [above of=Abeta2] {$A$};
    \node[blue,transition] (Ta) [right of=A, label=$t_A$] {$T_A$};
    \node[dotted,blue,place] (B0) [right of=Ta] {$B_0$};
    
  \path
  		(I) edge[post] node {} (Aalpha)
  		(Aalpha) edge[post,red] node {} (Talpha)
  		(Talpha) edge[post,red] node {} (A)
  				 edge[post,red] node {} (Abeta2)
  		(A) edge[post] node {} (Ta)
  		(Ta) edge[post] node {} (B0)
  		
  		(Abeta2) edge[post,green] node {} (Tbeta)
  		(Tbeta) edge[post,green] node {} (Balpha)
  		
  		(Balpha) edge[post,red] node {} (Talpha2)
  		(Talpha2) edge[post,red] node {} (B)
  		 edge[post,red] node {} (Bbeta)
  		(B) edge[post] node {} (TB)
  			edge[post] node {} (O)
  		(TB) edge[post] node {} (AFake)
  		(Bbeta) edge[post,green] node {} (Tbeta2)
  		(Tbeta2) edge[post,draw=green!75,bend left] node {} (Aalpha)
  		;
  		
  \end{scope}
\end{tikzpicture}

\caption{Vérification du fonctionnement du déplacement dans le cas des boucles.}
\label{fig:verifBoucles}
\end{figure}


\subsection{Vérification dans le cas de la duplication}
\label{section:synchroPetri}
Lors d'une duplication, il est nécessaire de procéder à un recollage par la suite des différentes branches. On reprend le schéma de la figure \ref{fig:deplacementForm2}, mais on décide en plus de dupliquer le segment $B(B_1, T_B, C_1)$ sur plusieurs machines.

Plusieurs approches sont possibles pour le recollement : 
\begin{itemize}
\item N'avoir qu'une place d'entrée pour le segment suivant, et relier la sortie des $\beta$-transitions dupliquées à l'entrée de la place du segment suivant.

Cela a pour conséquence d'avoir des jetons dupliqués pour la suite du parcours : le reste du réseau de Petri s'exécutera autant de fois qu'il y a eu duplication. Si ce comportement n'est pas désiré, il est nécessaire d'introduire une construction qui va supprimer les jetons superflus. 

On peut par exemple modifier la sémantique de manière à ce que la transition $T_C$ supprimera tout jeton arrivant dans ses places d'entrée si elle est en cours d'exécution, ou bien introduire un nouveau segment qui aura spécifiquement ce but, ou encore changer la sémantique de la place d'entrée $C$, de manière à ce que lorsqu'elle reçoit un jeton, elle ne tienne pas compte (et supprime) les $k$ prochains segments, ou $k$ est le nombre de machines sur lequel on duplique, moins $1$.

Cela peut par exemple servir à faire un effet de canon musical, et de partir au plus tôt pour la suite.

\item Avoir un segment tampon supplémentaire entre les $\beta$-segments dupliqués, et les segments suivants (représenté en orange sur la figure \ref{fig:duplicationEtRecoll}).
La conséquence est l'attente de la terminaison de tous les processus (au moins une fois, dans le cas de boucles à effet duplicant) sur toutes les machines.

Ce nouveau segment tampon (nommons-le $\gamma$-segment) devrait être situé sur le nœud du segment qui le suit.
\item Ne pas avoir de recollement : seule une branche continuera sur la suite et les autres seront mortes. Il revient au compositeur de choisir quelle branche aura la priorité.

Dans ce cas, on n'effectue la seconde opération de déplacement qu'avec la branche qui va effectivement être reliée à l'extérieur.
\end{itemize}

\begin{figure}[h!]
\centering
\input{formalisation/figures/duplicationEtRecoll}
\label{fig:duplicationEtRecoll}
\end{figure}


%%%%%%%%%%%%%%%%%%%%%%%%%%%%%%%%%%%%%%%%%%%%%%%%%%%%%%
%%%%%%%%%%%%%%%%%%%%%%%%%%%%%%%%%%%%%%%%%%%%%%%%%%%%%%

\section{\scshape Implémentation}
\begin{frame}{Implémentation}
	Plusieurs aspects : 
	
	\begin{itemize}
		\itemar Implémentation des algorithmes sur réseaux de Petri.
		\itemar Tests.
		\itemar Implémentation dans le logiciel i-score.
		\itemar Portage sur plate-formes embarquées.
	\end{itemize}
	
	L'implémentation dans le logiciel doit offrir des concepts simples aux compositeurs (abstraction de la technique).
	
	\begin{itemize}
		\itemar Création de concepts de haut-niveau travaillant avec l'\textsc{API Score} : Groupes, Clients, Permissions.
	\end{itemize}
\end{frame}

\subsection{Logiciel de test}
\begin{frame}{Logiciel de test}
	
\end{frame}

\subsection{API OSSIA}
\begin{frame}{Implémentation dans l'API OSSIA}
\end{frame}

\subsection{Résultats}
\begin{frame}{Résultats}

\end{frame}


\include{resultats}
\section{\scshape Perspectives}
\begin{frame}
	lolol
\end{frame}

\section{\scshape Conclusion}
\begin{frame}
	. Continuation en thèse -> intégration des notions spatio-temporelles.
	
	
\end{frame}

\bibliographystyle{alpha}
\bibliography{stagepetri}
\end{document}
